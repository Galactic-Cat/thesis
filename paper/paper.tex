\documentclass{paper}

\usepackage{adjustbox}
\usepackage{appendix}
\usepackage[english]{babel}
\usepackage{etoolbox}
\usepackage{framed}
\usepackage{hyperref}
\usepackage{amsmath, amssymb}
\usepackage[a4paper,margin=1.5in]{geometry}
\usepackage{graphicx}
\usepackage[utf8]{inputenc}
\usepackage{multicol}
\usepackage{wrapfig}
\usepackage[x11names]{xcolor}
\usepackage{listings}
\usepackage{mathtools}
\usepackage[x11names]{xcolor}

\hypersetup{
    colorlinks,
    linkcolor={green!50!black},
    citecolor={cyan!50!black},
    urlcolor={blue!80!black}
}

\setlength{\parskip}{1em}
\setlength{\parindent}{0pt}

\newcommand{\cn}{\textsuperscript{\color{red}[cite me]}}
\newcommand{\cc}[1]{{\color{cyan}#1}}
\newcommand{\df}{\coloneqq}
\renewcommand{\|}{\ |\ }
\newcommand{\tran}{^{\mkern-1.5mu\mathsf{T}}}

\lstnewenvironment{quicklst}[1][]
    {\lstset{float=tbhp,frame=single,captionpos=b,mathescape=true,escapeinside={!*}{*!},moredelim=**[is][\color{red}]{@r}{@},#1}}
    {}

\lstnewenvironment{haskell}[1][]
    {\lstset{
        float=tbhp,
        frame=single,
        captionpos=b,
        columns=fullflexible,
        language=haskell,
        mathescape=true,
        escapeinside={!*}{*!},
        keywordstyle=\color{Blue4}\bfseries,
        basicstyle=\small\ttfamily,
        commentstyle=\footnotesize\color{Green4}\normalfont\itshape,
        keepspaces=true,
        showstringspaces=false,
        numbers=left,
        numberstyle=\tiny,
        #1}}
    {}

\title{Maintaining parallelism in reverse-mode automatic differentiation on functional parallel array languages}
\author{Simon van Hus\\6147879\\s.vanhus@students.uu.nl}
\date{\today}

\begin{document}
    \maketitle

    \begin{abstract}
        In this paper we set out to make a simple reverse-mode automatic differentiation (AD) algorithm, that uses tracing for the forward pass, and preserves data parallelism in the reverse pass.
        To do this, we first try to formalize the notion of tracing somewhat.
        We find that while some flexibility in the definition of is needed for it to work well, we can also boil it down to picking a subset of data types to keep in the trace.
        We also define a couple of logical assertions that further help us in showing whether a trace does really contain the information that we need.
        Having defined tracing, in theory, but also over a Haskell DSL, we continue to automatic differentiation.
        Here we expand the tracing function into a forward-pass function by adding reference counting and intermediate values.
        Using this forward-pass trace as a map, we then show how we can do the reverse-pass.
        We also show that we can keep data-parallelism intact for the map and fold (reduce) operations.
        Finally, we also highlight how task parallelism can be used in the reverse-pass to possibly unlock even more efficiency.
    \end{abstract}

    \section{Background}
    \subsection{Automatic Differentiation} \label{sec:bg_ad}
        Automatic Differentiation (AD), like the name suggests, involves programmatically finding the derivative of some programmed function \cite{margossian2019review}.
        The other main method for programmatically finding the derivative of a function is numerical differentiation, which uses the finite difference method.
        By adjusting the input(s) to the function by a very small number, we can see the effect on the output(s) of the function.
        Unfortunately, due to the way real numbers are represented using floating-point computation, this method is prone to round-off error (or truncation error).
        AD avoids this by actually performing the differentiation on a program, to produce the differentiated program.
        This is very similar to how a human would differentiate a mathematical function (sometimes called symbolic or manual differentiation), but performed on a computer program.
        
        AD makes very explicit use of the chain rule of partial derivatives of compound functions, which provides a method for finding the derivative of compound functions and states that we can combine partial derivatives of parts of the function together into the complete derivative.
        Say we have some single-variate function $h(x)$, which is the compound function of the functions $f$ and $g$:
        \[h(x)=(f\circ g)(x)\]
        In this case, the chain rule tells us that the derivative of $h(x)$ is given by as $\tfrac{d}{dx}h(x)$:
        \[\frac{d}{dx}h(x)=\frac{d}{dx}(f\circ g)(x)=\frac{df}{dg}\biggr|_{g(x)}\cdot\frac{dg}{dx}\biggr|_x\]
        For clarity, in Lagrange's notation, where $h'(x)$ is the derivative of $h(x)$, this same statement can be expressed as:
        \[h'(x)=(f\circ g)'(x)=f'(g(x))\cdot g'(x)\]
        The chain rule also extends to compositions of more than two functions.
        For example, say we have a function $k(x)$ as below:
        \[k(x)=(f\circ g\circ h)(x)\]
        We can then find the find the derivative of $k(x)$ using the chain rule as well:
        \[\frac{d}{dx}k(x)=\frac{d}{dx}(f\circ g\circ h)(x)=\frac{df}{dg}\biggr|_{g(h(x))}\cdot\frac{dg}{dh}\biggr|_{h(x)}\cdot\frac{dh}{dx}\biggr|_x\]
        Again for clarity, in Lagrange's notation this would be:
        \begin{align*}
            k'(x)=(f\circ g\circ h)'(x)&=f'((g\circ h)(x))\cdot(g\circ h)'(x)\\
            &=f'((g\circ h)(x))\cdot g'(h(x))\cdot h'(x)
        \end{align*}

        The chain rule also provides us with a method of deriving multivariate functions.
        For instance, we can imagine a function $f(x,y)$.
        Now, the derivative of $f$ changes depending on which variable we wish to derive with respect to.
        Furthermore this is not a composition of functions, so the chain rule does not come into play.
        However, if we image the variables $x$ and $y$ as single-variable functions $x(t)$ and $y(t)$ we can find the derivative of $f$ with respect to $t$ using the chain rule.
        We get:
        \[f(x(t),y(t))\]
        Now to calculate the derivate of $f$ with respect to $t$, we first need to find the derivative of $x$ with respect to $t$ and the derivative of $y$ with respect to $t$.
        The chain rule tells us that the derivative of $f$ here is equal to the partial derivative of $f$ with respect to $x$ summed with the partial derivative of $f$ with respect to $y$.
        We can express this as:
        \[\frac{d}{dt}f(x(t),y(t))=\frac{\partial f}{\partial x}\biggr|_{x(t)}\cdot\frac{dx}{dt}\biggr|_t+\frac{\partial f}{\partial y}\biggr|_{y(t)}\cdot\frac{dy}{dt}\biggr|_t\]

        An important thing to note about the chain rule is that we still need the intermediate primal values in a compound function.
        Review the following compound function:
        \[(f\circ g\circ h\circ k)(x)\]
        In Lagrange's notation, the derivative becomes:
        \begin{align*}
            (f\circ g\circ h\circ k)'(x)&=f'((g\circ h\circ k)(x))\cdot g'((h\circ k)(x))\cdot h'(k(x))\cdot k'(x)\\
            &=f'({\color{green}g({\color{red}h({\color{blue}k(x)})})})\cdot g'({\color{red}h({\color{blue}k(x)})})\cdot h'({\color{blue}k(x)})\cdot k'(x)
        \end{align*}
        See how on the second line we have highlighted the primal parts of the equation, the intermediate values that we need for finding the derivative.
        Also note how values deeper in the chain are used multiple times; $h(k(x))$ is used twice: first in the derivative $f'$ and second in the derivative $g'$.
        $k(x)$ is even used three times.
        Looking at this example, it becomes very clear that it would be more efficient to calculate $k(x)$ once and save that result somehow, rather than recalculating it every time it came up.
        The storing and reusing of intermediate values is a fundamental property of AD, and is called ``sharing''.

        To actually implement automatic differentiation, we seek to break the target program down to its most basic mathematical operations, for which we know the derivatives.
        Then we can use the chain rule to combine them together into the derivative of the whole program.
        There are two main ways to actually resolve these derivatives: using either forward accumulation or backward/reverse accumulation.
        When applied in AD implementations these are commonly respectively referred to as forward-mode and reverse-mode.
        Both methods are described in the 1986 paper ``The arithmetic of differentiation'' by B. Rall \cite{rall1986arithmetic}.
        
        In forward-mode AD we move through the program to differentiate in normal execution order.
        By knowing which input variable we wish to differentiate, we can compute every step of the derivative as our inputs are used by the program.
        Rall demonstrates this using a method known as dual-numbers, where each real number is represented by a pair of numbers, similar to complex numbers.
        In dual-numbers, the first number in the pair represents the primal part of the number, whereas the second number represents the derivative part (called the tangent in forward-mode).
        When we compute with these numbers through arithmetic operations, we can operate on the primal parts as normal, and use derivative rules to calculate the derivative of the result using the tangent parts.
        An example of this is given in Equation \ref{eq:dualnumbers}, where $\dot{a}$ is the tangent part of some real number $a$.
        \begin{equation} \label{eq:dualnumbers}
            (a,\dot{a})\cdot(b,\dot{b})=(a\cdot b, \dot{a}\cdot b+\dot{b}\cdot a)
        \end{equation}
        Now we can find the derivative of some program with regards to the input $x_i$ by setting $\dot{x_i}$ to $1$, setting the tangents of all other inputs to $0$, and just running through the program calculating tangents as we go.
        The tangent part of the output value(s) is also the calculated derivative of the whole program.

        While forward-mode AD is fairly straightforward, it comes with some drawbacks.
        The main one being that for a function $f:\mathbb{R}^n\to\mathbb{R}^m$ with $n$ inputs and $m$ outputs, to get the effect of each input variable on each output variable, we would need to perform $n$ passes over the function, one for each input variable (or we need to track $n$ tangent parts for each step).
        This is can be cumbersome, especially if $n$ is much larger than $m$.
        For those cases, we might be better off with reverse accumulation, or reverse-mode.

        In reverse-mode, we peg the derivative part of one of our outputs with some seed (often $1$), and set the derivative parts of the other outputs to $0$.
        These derivative parts are generally referred to as adjoints instead of tangents in reverse-mode.
        When the outputs are set, we can work our way back through the function, calculating the derivative parts from the output to the input.
        Intuitively, this computes the gradient of the output dimension we pegged to $1$, or the direction of the steepest slope.
        Practically, the idea of working back through a program requires some way of knowing where the outputs came from (a sort of dependency structure).
        This then requires a forward pass, to find this structure, to calculate the intermediate values, and often to setup any dual-numbers or other implementation details.
        And while reverse-mode is definitely harder to implement, it also provides us with a way to calculate the sensitivity of all inputs to an output, which is much more efficient for functions with many more inputs than outputs (which can be quite common in certain applications like neural networks).

        In mathematical terms, calculating the partial derivative of one output with regards to one input, means calculating one cell in the Jacobian, the matrix of all partial derivatives.
        For a function $f:\mathbb{R}^n\to\mathbb{R}^m$ with $n$ inputs and $m$ outputs, the Jacobian $J_f$ would be a $n\times m$ matrix.
        Here a column $i$ represents the partial derivatives $\tfrac{\partial\vec{f}}{\partial x_i}$, where $\vec{f}$ are all outputs of $f$, and $x_i$ represents a single input.
        A row $j$ then represents the derivatives $\nabla f_j=\tfrac{\partial f_j}{\partial\vec{x}}$, where $\vec{x}$ are all inputs of $f$, and $\nabla f_j$ is also known as the gradient of the single output value $f_j$.
        This is also shown in Equation \ref{eq:jacobian}, showing the Jacobian for some function $f$ with $n$ inputs ($x_1,\dots,x_n$) and $m$ inputs ($f_1,\dots,f_m$).
        An important take-away here is that forward-mode computes the derivatives of all outputs with regards to a single input, so a column in the Jacobian, and reverse-mode computes the derivatives of all inputs with regards to a single output, so a row in the Jacobian.
        Again, if we want to calculate the full Jacobian, forward-mode is more efficient when we have more outputs than inputs or when the Jacobian has more columns than rows, and the reverse-mode is more efficient for functions with more inputs than outputs or for Jacobians with more rows than columns.

        \begin{equation} \label{eq:jacobian}
            J_f=\left[\frac{\partial \vec{f}}{\partial x_1},\dots,\frac{\partial \vec{f}}{\partial x_n}\right]=\begin{bmatrix}\nabla f_1\\\vdots\\\nabla f_m\end{bmatrix}=\begin{bmatrix}
                \frac{\partial f_1}{\partial x_1} & \dots & \frac{\partial f_1}{\partial x_n}\\
                \vdots & \ddots & \vdots\\
                \frac{\partial f_m}{\partial x_1} & \dots & \frac{\partial f_m}{\partial x_n}
            \end{bmatrix}
        \end{equation}

        While it has long been known that reverse-mode automatic differentiation could be executed in time equal to some constant multiple of the execution time of the primal program \cite{linnainmaa1976taylor}, it seemed that a constant multiple of the execution memory was also needed, which could become very expensive for large programs.
        However, in 1992, Andreas Griewank showed that by using taping and checkpointing we could trace time complexity for space complexity to reduce either to a constant multiple of the log of the execution time \cite{griewank1992achieving}.
        In general the practice of taping refers to a form of tracing on the program we wish to differentiate, where we execute the program as normal and record all the steps and intermediate values in a first-in-last-out data structure referred to as a ``tape'' or Wengert list.
        In a second phase to the reverse-mode algorithm, the tape is then used to calculate the derivatives in question, which due to the first-in-last-out nature of the tape, is in the precise reverse of the execution order of the program.
        An important advantage of taping is that by giving each variable and intermediate calculation a unique id we can avoid redundant execution, because we can just refer to the intermediate value or tangent/adjoint stored in the tape.
        While taping is efficient time-wise, it clearly adds a memory overhead that can be quite sizable for large programs.
        Checkpointing aims to address this by storing multiple parts of the tape to memory attached to checkpoints in the program's execution.
        The trick here being, that on the reverse-pass only the intermediate values from the most recently encountered checkpoint are loaded from memory, intermediate values that were not stored as part of this checkpoint are recalculated.
        By strategically placing these checkpoints, and deciding which intermediate values are stored, this can cut the size complexity at a relatively small time complexity increase.
        It should be noted that automatic differentiation can also be performed on a program where we do not have any specific inputs.
        We can do this using source transformation \cite{bischof2000computing}.
        In its most basic form source transformation can be implemented as just interlacing the derivative calculations into the regular program.
        An example of this is provided in Listing \ref{lst:transformation_fw}, where we calculate the derivative of some variable y (as dy) with regards to the variable x1.
        For reverse-mode AD this kind of interlacing is not possible, as we need to reach the end of the program before we can start the reverse pass, which is exactly why we record our steps on the tape: so we can reverse over the tape and know how to produce our reverse AD program.
        An example of this is provided in Listing \ref{lst:transformation_rv}, where we again calculate the derivative of y (as dy), with regards to x1 and x2.
        So, to summarize, for a function $f:A\to B$, source transformation finds the derivative function for any input in the domain $A$, whereas dual-numbers (or similar approaches) find the derivative function for a specific input $a\in A$.
        Of course, in complex functions with a lot of control flow, source transformation can become cumbersome as it needs to account for all possible inputs, whereas dual-numbers only needs to account for one.

        \begin{quicklst}[caption={An example of forward mode AD by source transformation, with the AD statements in red}, label=lst:transformation_fw, gobble=12]
             x1 = 15
            @rdx1 = 1@
             x2 = 7
            @rdx2 = 0@
             r1 = x1 + x2
            @rdr1 = dx1 + dx2@
             y  = r1 $\times$ x2
            @rdy  = r1 $\color{red}\times$ dx2 + dr1 $\color{red}\times$ x2@
        \end{quicklst}

        \begin{quicklst}[caption={An example of reverse mode AD by source transformation, with the AD statements in red}, label=lst:transformation_rv, gobble=12]
             x1 = 15
             x2 = 7
             r1 = x1 + x2
             y  = r1 $\times$ x2

            @rdy  = 1@
            @rdr1 = dy $\color{red}\times$ x2@
            @rdx2 = dy $\color{red}\times$ r1 + dr1 $\color{red}\times$ 1@
            @rdx1 = dr1 $\color{red}\times$ 1@
        \end{quicklst}
        
        For forward-mode AD, the evaluation of the derivative is done during execution.
        Like in 1996's FADBAD package, which provided both forward-mode and reverse-mode AD for C++ \cite{bendtsen1996fadbad}.
        The reverse-mode uses the taping method described by Griewank, implemented through a method called operator overloading.
        In forward-mode, operator overloading refers to providing the basic mathematical operators with methods that work on the numbers represented by a pair (of a primal part and a tangent part); this is the dual-numbers approach we mentioned before.
        For reverse-mode, operator overloading is used to rewrite the basic mathematical operators, so they record their use and intermediate values to a single tape data structure.
        A similar implementation was also provided by Griewank et al. in the 1996 package ADOL-C \cite{griewank1996algorithm}, again in 2001 using more efficient expression templates by Aubert et al. \cite{aubert2001automatic}, and later in 2014 by Robin Hogan \cite{hogan2014fast}.

        Source-code transformation is eventually also implemented, in the Tapenade AD program \cite{hascoet2013tapenade}.
        Tapenade adds derivative calculations to the code, but also employs lazy/delayed evaluation in the forward pass.
        This allows Tapenade to do some activity analysis, which in turn allows it to combine or discard some partial derivatives to be more efficient.
        It also implements the previously discussed checkpointing, where part of the tape is stored to be restored and differentiated later.
        This, in theory, allows for differentiating programs of arbitrary size, because the differentiation process is not limited by the size of the working memory \cite{griewank2008evaluating}.

        More recently implementations, like Fei Wang et al. 2019 paper, have shown how to simplify reverse automatic differentiation using continuation passing style and delimited continuations \cite{wang2019demystifying}.
        This method uses dual numbers and cleverly overloads operators so they call the forward pass as a continuation and then perform the backwards pass on the returned value.

        In 2022, Krawiec et al. show how reverse-mode AD can be extended efficiently to higher-order functional programs \cite{krawiec2022provably}.
        While the Wang paper also did this, Krawiec uses the functional nature to provide a correctness proof of the reverse-mode AD, something that had previously only been done on implementations that were either asymptotically inefficient or only worked on first-order languages.
        They do however need taping again to make it provable and efficient.\\
        Vákár and Smeding provide a provably correct form of higher-order reverse AD without taping in their 2022 paper \cite{vakar2022chad}, based on earlier work by Elliott in 2018 \cite{elliott2018simple}.

        And in 2022 as well, Schenck et al. show how to do both forward-mode and reverse-mode automatic differentiation on second-order array language with nested data parallelism \cite{schenck2022ad}.
        They do this by eliminating taping again, which forces sequential execution, by allowing potential redundant execution.
        But by limiting their AD implementation to second-order functional languages, they can largely avoid this redundancy with efficient program transformations on parallel operators.

        Finally, in 2023, Smeding en Vákár bring back explicit dual-numbers to reverse AD \cite{smeding2023efficient}.
        However, instead of pairing each number with its computed adjoint, they instead pair it with a linear backpropagator function, which they can then later chain to get the full derivative.
        While this initially seems to eliminate the need for taping, they find that through optimizations they return to a concept that is very close to taping and show that it is in fact equivalent.

    \subsection{Tracing}
        Tracing is a concept in computer science that is often left without proper definition.
        While the main ideas behind tracing are well known, they are generally assumed known by the reader and therefor left without explanation.
        This is also in part because, in software engineering, the term tracing also refers to finding the origin of some call (``tracing'' the call stack), which is only tangentially related to the tracing we are interested in, but can leave definitions of tracing a bit muddled.
        This is why, in Section \ref{sec:tracing}, we will discuss more about that proper definition.
        For now, it is important to know that, when we refer to tracing in this paper, we speak about tracing the path of computation through a program, given some (valid) input to said program.
        In other terms, given a program and an input, we walk through the program and record each computational step for some later purpose, like automatic differentiation.
        This recording can happen with some domain-specific pseudo-language, or in full fledged code if we wish to revaluate the trace later (or a combination of the two).

        Doing tracing gives us some interesting insights into a program we trace.
        First, it effectively ignores control flow.
        This is fairly intuitive, when given a set of inputs to a program, the control flow will control what path the program uses, and since we only record computations we find what is often dubbed a ``straight-line program'' for some inputs.
        This can be useful for instance in automatic differentiation, where we often only want to differentiate a computation, not the entire program including unused branches.
        This will also be the use of tracing in this paper, as laid out in Section \ref{sec:ad}.

        As mentioned, in literature we see this type of tracing used for automatic differentiation.
        One of these uses was by Bischof in 1991 \cite{bischof1991issues}.
        In his paper, Bischof discusses the use of the computational graph of a program in automatic differentiation (using ADOL-C \cite{griewank1996algorithm}).
        The computational graph of a program is an directed acyclic graph, where each node contains a computational step in the program, and edges connect these steps in the execution order of the program.
        Bischof creates this graph from the tape produced by ADOL-C, which makes sense: for automatic differentiation as discussed, the tape acts as a sort of trace, recording the steps that are important in the automatic differentiation.
        Bischof then uses a graph colouring algorithm on the computational graph to highlight ``component functions'' that may be differentiated concurrently, as to improve the running time of the algorithm.
        In 2008, Bischof et al. expand on this by extending the tracing automatic differentiation to loops \cite{bischof2008parallel}.
        They do this by extending ADOL-C, paying specific attention the parallelization opportunities present in automatic differentiation.

        In a similar vein, Dougal Maclaurin presented in his PhD thesis in 2016 \cite{maclaurin2016modeling} a paper introducing Autograd.
        A software package to automatically differentiate Python code (including AD for the vector library Numpy).
        As Python is an expressive JIT-compiled (Just In Time) dynamic language, they opt for tracing to construct the computing graph on the fly when a function is called, and like Bischof's work this allows them to do the backwards pass off reverse-mode AD on the computational graph.
        They do this by wrapping their variables as "nodes" in the computational graph.
        When a variable is used, it is first unwrapped for use, and then the result of whatever operation used the variable, is stored as a new variable and wrapped as well.
        The original variable and the produced variable are then linked such that the produced variable stores a reference to the original variable.
        This creates the reverse computational graph, which is exactly what is needed for the reverse AD pass.

        TensorFlow, a machine learning library, also uses tracing to create computational graphs \cite{abadi2016tensorflow}.
        This kind of tracing is not as low to the ground as actually following individual computations.
        Since TensorFlow mainly focusses on building artificial neural networks, the computational graph is made explicit by the programmer.
        While there are some nuance differences between a computational graph of a neural network and the neural network itself, these differences are somewhat unimportant.
        More interestingly, TensorFlow allows for the partial execution of the computational graph.
        While Bischof's use of a graph colouring algorithm already suggested this, TensorFlow actively uses this technique to re-run partial computational graphs, which works well for the explicit nature of neural networks, as the computational graph stays unchanged even if the inputs change (as neural networks do not have internal control flow).

        Finally, 2018's JAX uses tracing to enhance performance of general machine learning code \cite{frostig2018compiling}.
        The programmer annotates functions to be analysed by JAX, which then traces as optimizes them.
        Rather than finding the computational graph (or predefining it), JAX waits for Python to execute the function and actually traces it.
        Then, JAX optimizes it, mainly through a process called fusion, which is discussed in Section \ref{sec:bg_array}.
        This is also where JAX gets its name: Just After eXecution, as it waits for Python to execute the function first.
        It should be noted that JAX can only do this for functions which are pure-and-statically-composed (PSC), meaning functions that have no side-effects and that do not change with different inputs.
        Again, machine learning code is especially suited for this, as it often already satisfies this PSC assumption.
        
    \subsection{Functional Parallel Array Programming} \label{sec:bg_array}
        Array programming languages are programming languages that treat the array as a central data structure.
        This generally includes that functions, both user-defined and built-in, could be applied to arrays through vectorization.
        Vectorization involves applying a function to every element of an array at the same time.
        For instance, vectorization of addition would add two arrays together element-wise.
        This is shown in Equation \ref{eq:vectorization}, where $\vec{a}$ and $\vec{b}$ both are arrays of the same size.
        \begin{equation} \label{eq:vectorization}
            \vec{a}+\vec{b}=[a_1+b_1,\dots,a_n+b_n]\text{ iff }|\vec{a}|=|\vec{b}|
        \end{equation}
        In general vectorization would only work for arrays of the same size were it not for another central concept: broadcasting.
        Broadcasting involves the resizing of arguments to functions so they can be used.
        A very clear example would be if we wished to add a scalar value to each element in an array of scalars.
        To do this with vectorization alone would mean we'd need another array which replicates the scalar we wish to add for each element in the array we wish to add it to.
        Broadcasting basically does this for us, as exemplified in Equation \ref{eq:broadcasting}.
        \begin{equation} \label{eq:broadcasting}
            \vec{a}+2=[a_1+2,\dots,a_n+2]
        \end{equation}
        
        Array programming languages also often support higher-order operators for use on arrays.
        An important operator for arrays is fold (or reduce), which applies a binary function to elements in an array, where one argument accumulates the previous results.
        It is easy to imagine how such an operator could be used to, for instance, sum all the items in a 1-dimensional array.
        An important realization is that, since fold only returns the final result, fold can reduce the dimensions of an array by one.
        In our summation example, we fold a one-dimensional array into a zero-dimensional array, namely a scalar value.
        Similar to fold is scan, which like fold applies a cumulative binary function to each element in the array, but rather than returning only the result, it returns all intermediate results in an array (with the last element being the final result).

        Other important array functions include map, which applies a function to each element in an array.
        Then, forward permutation (scatter) and backwards permutation (gather), which permute one array into a new one by respectively mapping the indices of the source array to those of the new array or the indices of the new array to those of the source array.
        Generate, which generates a new array as well, but does by taking the dimensions of the desired array and a function that takes in an index and outputs a value.

        We should also not gloss over the actual implementation of these arrays, especially in functional languages where there exists two major ways of constructing arrays \cite{svensson2014defunctionalizing}.
        Pull-arrays are the more used of the two, here arrays are represented with a function from an index to a value.
        In push-arrays, consumers are provided with a method to write into memory.
        This means that the way that the efficiency of array operations can change based on the array representation.
        For instance, indexing is faster on pull-arrays while push-arrays are quicker to concatenate.
        This basically divides the array operations in two camps: push-operations and pull-operations.
        
        These two camps play an important role in a concept of fusion.
        When we have multiple back-to-back parallel array operations, executing them naively introduces a lot of overhead for reading and writing intermediate values to memory.
        Instead fusion allows us to combine these operations together, so we can compute them in one go without the overhead of storing intermediates.
        However, we cannot just go chaining parallel operations, not all parallel operations fuse together nicely.
        In fact, pull-array operations only fuse with other pull-array operations, and the same goes for push-array operations.
        This means for instance that we can fuse multiple scatter operations, but not a scatter and gather operation.

        Now the reason for choosing a (functional) array language over a general language is often because we need to process large amounts of numerical data, and arrays are well-suited for parallelism.
        To be precise, we are talking about data parallelism here.
        Task parallelism is when two or more computer processes run simultaneously on different processor cores.
        Data parallelism is when an operation (or a string of operations) is done element-wise on data structure like an array.
        The parallelism of data parallel processing of these operations on each element, rather than the parallelism of different processes.
        An important distinction between task and data parallelism, is that while parallel threads in task parallelism can generally start, run, and end independently of each other, data parallelism threads move in lockstep with each other.
        This is lockstep or synchronous execution means that the execution does not continue until the current operation has been applied to all elements in the array, which may be important if we want to do multiple parallel operations back-to-back.
        Furthermore, modern GPU architectures are especially well-suited for this type of synchronous parallelism, as graphics processing overlaps in large part with parallel array processing.

        A good starting point for the history of functional parallel array programming was in 1992, with G. Belloch's paper on the parallel array programming language NESL \cite{blelloch1992nesl}.
        The language was strongly-typed and had no support for side-effects, making it a functional language.
        The main way to add parallelism was through the inherently data-parallel ``vectors'' the language introduces in lieu of lists.
        These vectors could also be nested, and functions could run in nested parallel on these vectors.
        Another major inclusion was to allow user-defined functions to be run (in parallel) on these vectors, making it possible to write more complex nested data-parallel algorithms than before.

        The functional language Haskell, saw the introduction of task-parallelism well before its first official release, through libraries like pH \cite{maessen1995semantics}.
        Some data-parallelism followed \cite{hill1995data, herrmann1999parallelization, ellmenreich2000application}, but this was limited to applying a function over a flat array.
        However, in 2001 nested data-parallelism was introduced to Haskell by the NEPAL project by Chakravarty et al. \cite{chakravarty2001nepal}.
        The paper largely focusses on reimplementing NESL as a Haskell library, but creates a much more expressive data-parallel language doing so.
        This is because NESL was rather limited in scope, whereas Haskell was already a fully-fledged functional programming language.
        Two important concepts come to the forefront in the NEPAL paper, namely flattening and fusion.
        Both in NESL and in NEPAL, higher-dimensional nested parallelism is ``flattened'' to a single distributed parallel operation.
        In NESL, this meant that data-types had to be limited to tuples and the vectors it introduced, to make sure this flattening operation worked correctly.
        Since then however, Keller and Chakravarty had shown this flattening transformation could also be applied more generally to cover the full range of types of general programming languages \cite{keller1999transformation,keller1998flattening,chakravarty2000more}.
        This allowed them to apply the nested data parallelism of NESL to a more expressive language Haskell with NEPAL.
        Furthermore, they also showed that in combination with fusion it could produce efficient code for distributed machines \cite{chakravarty2001functional}.
        Fusion is where multiple separate parallel operations are combined into a single parallel operation, which greatly improves performance of complicated parallel programs.
        This is important because many operations on arrays introduce the need for intermediate arrays to be computed.
        Doing this in parallel leads to more problems, as these implementations rely on gang parallelism, where the parallel threads remain in lockstep with each other \cite{feitelson1996packing}.
        Fusion helps us here, as we can reduce the number of intermediate arrays to be generated, as we can calculate the results of multiple operations at once \cite{keller1999distributed,chakravarty2007data}.
        
        All this work culminated in 2007's Data Parallel Haskell (DPH) \cite{peyton2008harnessing}, by Peyton-Jones et al.
        Its main feature was the parallel array, that like NESL's vectors, was the main way of adding parallelism to a program.
        However, these parallel arrays could now hold any type, such as other arrays or functions, like Haskell's native (non-parallel) lists.
        Furthermore, DPH provides parallel variants of Haskell's native list functions, and a parallel alternative to Haskell's list comprehensions.
        The main difference between Haskell's native lists and DPH's parallel arrays (besides the parallelism) was that evaluating any value in a parallel array would require evaluation on all the array's elements, whereas Haskell as a lazy language would not normally do that.
        This is to be expected, as parallelism becomes meaningless if it is only applied to a single entry of an array.

        Outside of Haskell, a functional array-programming dialect of C was developed: Single Assignment C (SAC) \cite{scholz1994single, scholz2003single, grelck2005generic}.
        It would go on to distinguish itself as a functional array programming language in a style more familiar to programmers of imperative languages (like C).
        The main mechanic in SAC is the with-loop, which takes a generator that dictates a looping mechanism and an operation that dictates the return value.
        These operations can be functions like ``fold'' to reduce the rank of an array, or ``genarray'' to generate new (multidimensional) arrays.
        Besides the imperative style, the main draw of SAC is that its performance is comparable to Fortran and C, while its programs are generally more concise (for intensive numerical applications.)

        In 2010, Keller, Chakravarty, et al. presented a new data-parallelism approach for Haskell in ``Regular, Shape-polymorphic, Parallel Arrays in Haskell'' \cite{keller2010regular}.
        Previous approaches had focussed on irregular arrays, where on array could contain arrays of different lengths.
        The library Repa, introduced in this paper, was made for regular arrays where arrays of each nested rank are the same size.
        However, this allows the library to be purely functional and support shape polymorphism.
        While DPH was purely functional as well, it wasn't especially performant on regular arrays and it also did not support shape polymorphism.
        In shape polymorphism, the type of a collection is fixed (unlike in type polymorphism), but the shape of the collection is not \cite{jay1994shapely}.
        For instance, under shape polymorphism a function may be applied to either a flat array, or a 10-dimensional one.
        While shape polymorphism for functional arrays had been implemented before in SAC, Repa implemented it by embedding it into Haskell's type system, whereas the SAC implementation had required a purpose-built compiler.
        This also allowed programmers to more easily see and control the shapes of their multidimensional parallel arrays, and build their own shape polymorphic parallel functions.

        In 2011, Repa was succeeded by the Accelerate project \cite{chakravarty2011accelerating}.
        Accelerate is a library for Haskell, aimed specifically at bringing parallel array programming to modern GPUs.
        It mimicked many of Haskell's native list functions with parallel alternatives (that run on the GPU), and used the typed shaped polymorphism from Repa.
        It also separated ``collective'' (array) computations and scalar computation by wrapping these in Haskell monads.
        Here, collective computations could include scalar computations, but not the other way around.
        This meant excluding nested and irregular data parallelism, which in turn allows Accelerate to efficiently run on GPUs (which are much more constrained than CPUs).
        It also meant that these arrays could only contain scalars, no functions or other types.
        
        Another interesting example of a parallel array programming language is Remora by Slepak et al. \cite{slepak2014array}
        The language, inspired by earlier array programming languages APL \cite{iverson1962programming} and J implements rank-polymorphism.
        Rank polymorphism is similar to shape polymorphism, but it annotates functions and operators with an array rank they can operate on, and was also present in Repa and Accelerate.
        Remember that scalars are considered rank 0 arrays, a flat array is rank 1, a matrix is rank 2, et cetera.
        In rank polymorphism, arguments are transformed (re-ranked) such that they are the rank required for a specific function or operator.
        Specifically, an operator defined for a certain rank, is automatically defined for any higher rank, because it can be mapped over these higher dimensions.
        This is subtly different from the more general shape-polymorphism, as rank only refers to the number of dimensions, while shape also contains information on the size of these dimensions.
        With Remora, Slepak et al. tried to shed some light on the more ``murkier corners'' of the array-computational model.
        They do this by generalizing the array-computational model, which then allows them to both address some of the shortcomings of APL, but also allows them to extend the model to allow arrays of functions and arrays of arguments, which in turn allows for the parallel MIMD (multiple instruction, multiple data) architecture, rather than only SIMD (single instruction, multiple data) parallelism.

        In 2017, we got one of the major current functional data-parallel array languages in Futhark \cite{henriksen2017futhark}.
        Futhark's design focusses on efficient nested data-parallelism.
        They do this by using both ``aggressive'' fusion (fusing as much as possible), followed by flattening (like we saw in NESL).
        Finally through some more optimizations, Futhark produces very performant programs.
        To facilitate this performance however, they do not support higher-order programming, as Futhark only supports up to second-order.

        Finally, a more recent parallel array programming language is Dex \cite{maclaurin2019dex,paszke2021getting}.
        Rather than avoiding loops and explicit indexing, like NESL, NEPAL, DPH, and Repa had all done, Dex suggests that these features might introduce more clarity, if only they were implemented correctly.
        The main idea is to treat index sets as types and arrays as functions.
        In reality this ``index comprehension'' can also be seen as functions that return arrays, and allow declaring iteration over multiple dimensions in a single line.
        Of course, this is the same idea as pull-arrays, a representation also used by Accelerate under the hood.
        However, the main novelty of Dex is that they use this to make explicit loops, which in turn makes some parallelism opportunities also explicit.
        Also when these index comprehensions are presented back-to-back, opportunities for fusion become fairly clear as well.
        In their paper, they also show that on some benchmark problems, Dex performs similarly to Futhark, as a functional array programming language that was specifically designed to write performant parallel GPU code.


    \section{Tracing}
    In the broadest terms, when we trace a program, we track the most basic steps the program takes provided some input.
    This is relevant for many applications in Computer Science.
    For example, certain automatic differentiation (AD) effectively implement the forward-pass as tracing, and then perform the reverse pass on the trace\cn.
    Tracing is also used in Artificial Intelligence, where tracing applications can help determine how much memory needs to be allocated, which can speed up training if the model is run multiple times\cn.

    However, despite its ambivalence, tracing is rarely properly defined, or defined only for a specific use case.
    So, in this section we set out to create a more general definition of tracing.
    
    To start, it will help us along to set clear expectations for what we expect a tracing function to do.
    In the simplest terms, we expect a tracing program to take an input program with a set of inputs, and output a ``trace''.
    This output trace is generally defined as a set of operations the input program performed on the inputs.
    A term often used for an output trace is a ``single-line program''\cn: a program without control flow.
    Clearing control flow like if-then-else statements is only natural: after all, provided some input the program will only walk down one variation of this branching path.

    Furthermore, it is also generally accepted that the trace consists of a subset of the syntax of the input program.
    Because we are generally more interested in what happens to the data in our program, we can ``trace away'' functions and data structures.
    More precisely, say our input program has the types as defined in Equation \ref{eq:typebase}, where we have sum-types as $\tau+\sigma$, product types as $\tau\times\sigma$, functions as $\tau\to\sigma$, literal real numbers, and literal Booleans.

    \begin{equation}
        \label{eq:typebase}
        \tau,\sigma\coloneqq\tau+\sigma\|\tau\times\sigma\|\tau\to\sigma\|\mathbb{R}\|\mathbb{B}
    \end{equation}

    We can imagine our simplified language, in which we will express our trace -- as a language with fewer type formers.
    By choosing a subset of the type formers in our program, we can indicate which data structures should be traced away.
    A common option is to keep only ``ground types'', where we defined a ground type as a type that isn't constructed of other types.
    Looking at our example in Equation \ref{eq:typebase}, a trace keeping only these ground types would keep only the real numbers and the Booleans as they are not built of other types.
    Another common option is to keep only continuous types, tracing away all unground and discrete types.
    Doing that on our type set in Equation \ref{eq:typebase} would leave us with only the real numbers.
    This is under the assumption that the discrete types aren't actually used as data we're interested in tracing of course, but since tracing will remove all control flow from the program, keeping Booleans and operations on Booleans intact may be meaningless. 

    The main take-away here is that there is some freedom of choice in what to trace away.
    What parts we keep and what parts we trace away is very dependent on what information we want to keep in our trace, which in turn is dependent on what our exact goal is for the tracing in the first place.
    
    We can also choose to keep some of our unground types, but then we run into a problem.
    Say we keep only functions ($\tau\to\sigma$) and real numbers, but our input program contains a function with type $\tau\to(\sigma_1+\sigma_2)$.
    This typing is valid in our input program, but no longer valid in our trace, so we find ourselves in a bind.
    It will be impossible to trace away the sum-type in the output of the function without tracing away the function itself.
    This is because tracing something away basically means either deconstructing or ignoring it in the trace.
    For instance, tracing away a sum type like a tuple, would mean tracing the individual components of that tuple to trace it away.
    Whereas keeping things in the trace means just keeping them untouched.
    Therefor, we cannot keep a type like a function $\tau\to(\sigma_1+\sigma_2)$ in our trace, because we can't access the sum type without tracing away the function.
    Of course we could define a subset $\tau',\sigma'\coloneqq\mathbb{R}$ and then redefine (or add a definition for) our function so that it becomes $\tau'\to\sigma'$ making it safe to trace.
    This then underlines the rule at work here: we can only keep types that do cannot be constructed of types that are traced away.
    This is why the ground types are a natural set of types to keep, as they are never constructed from other types.
    In a similar vein, although slightly less clear, we also trace away any operators that perform on or produce types that we do not keep in our trace.
    For example, when we choose to keep only real numbers, we can trace away all Boolean values and operations that take in or produce Boolean values.
    After all, they will be meaningless in our limited trace language.
    This ``tracing away'' often comes down to rewriting or omitting these operations in the trace.
    For operations that influence control flow this is simple, however if we have types that we wish to trace away in some data that we use or (more importantly) combine with types that we do wish to keep, omission might mean our trace is missing steps.
    We might be able to express these types we trace away as types we keep, if they are necessary, but this might be more work than it is worth.

    It seems that our tracing definition comes down to a function that takes in a program and an input to that program, and outputs the steps taken by the program run on the input.
    Where the input program takes uses some set of types, of which only a subset is kept in the trace, where the types in this subset may not be constructed using types from outside of the subset.

    What now remains is a concrete definition of the output of the tracing program.
    We have already stated that it should somehow contain the steps done in by the input program.
    The steps we wish to record are generally basic operations like arithmetic operations.
    But other operations, such as operations on arrays, can also be added depending on the ultimate goal of the tracing.
    More importantly, as we expect our trace to be akin to a single-line program, we may consider our trace as a series of let-bindings, akin to A-normal form\cn.
    
    \subsection{Tracing Correctness} \label{sec:correctness}
        Before going into specifics on how to implement tracing, it would be a good idea to formalize when a trace is actually correct.
        To discuss this, we imagine we have some program with an any valid input, and a tracing function that given said program and input, returns a trace and an output to the program.
        Furthermore, we require the trace to be a program itself, which when run also produces the same output.
        We can denote this, for clarity, as in Equation \ref{eq:trace_eq}.
        \begin{equation}
            \label{eq:trace_eq}
            \text{trace}:\text{Program}\times\text{Input}\rightarrow\text{Trace}\times\text{Output}
        \end{equation}

    \subsection{Tracing Steps} \label{sec:steps}
        We now define some basic tracing steps for some arbitrary language.
        To do this we first define a language on which we will operate.
        We do this in Listing \ref{lst:language}, where we define a basic lambda calculus.

        \begin{quicklst}[caption=Basic language, label=lst:language, gobble=12]
            !*\textbf{Types:}*!
                $\sigma,\tau\coloneqq\mathbb{R}\|\sigma\to\tau$

            !*\textbf{Terms:}*!
                $s,t\coloneqq$
                    $r\in\mathbb{R}\quad$!*\text{(literal real numbers)}*!
                  | $\lambda(x:\tau).s\quad$!*\text{(abstraction)}*!
                  | $s\ t\quad$!*\text{(application)}*!
        \end{quicklst}

        Tracing through this lambda calculus is pretty straightforward.
        We will define our trace as some set $T$, such that our tracing function $\text{trace}(\Gamma,e)\to(v,T)$ gives us a value $v$ and a trace $T$ for some environment $\Gamma$ and some expression $e$.
        To not go into too much implementational overhead, we will ignore naming the items in the trace for now.
        However, for clarity we will define an evaluation function $\text{eval}(e)$, which will resolve an expression $e$ to its actual value.
        Listing \ref{lst:tracing} shows how tracing the lambda calculus in Listing \ref{lst:language} would look like, if we choose to trace away the function type (leaving us only with real numbers).
        We see that tracing a literal real number $r$, just adds $r$ to the trace, after all real numbers aren't traced away.
        Tracing application is also fairly straightforward: we trace the argument $t$, and then the function $s$ with $t$.
        Only our $\lambda$-abstraction is a little more complex.
        Since nothing actually happens our trace here is empty, however this does not mean that we can ignore the steps the abstracted expression takes, those we might still be interested in.
        So we can rewrite the lambda expression, to move the trace over, and trace only the body when it is actually applied.
        This also has as a side-effect on application, that we call the result of the trace of $s$ as a function, as that will be the rewritten lambda expression we got from tracing abstraction.

        \begin{quicklst}[caption=First tracing rules, label=lst:tracing, gobble=12]
            $\text{trace}(\Gamma,r\in\mathbb{R})\Rightarrow(r, \{r\})$

            $\text{trace}(\Gamma,\lambda(x:\tau).s)\Rightarrow(\lambda(x:\tau).trace(\Gamma\cup x, s),\emptyset)$

            $\text{trace}(\Gamma,s\ t)\Rightarrow(\text{eval}(s\ t),\text{trace}(\Gamma, s)(\text{trace}(\Gamma,t)))$
        \end{quicklst}

        It should be noted that tracing abstraction and application this way, effectively traces away function types.
        This is generally what we want from a trace, but it is somewhat meaningful to realize what would happen if we chose not to do that.
        In that case, we no longer deconstruct functions, so our trace on abstraction becomes similar to our trace on real numbers: we just return the function as we found it.
        As for application, we can no longer trace the body of our function, so all that remains in tracing its argument and perhaps denoting the application took place.

        Now to illustrate how tracing would eliminate control-flow, we would first need to add some to our language.
        We can extend our language in Listing \ref{lst:language} with an if-then-else statement, as shown in Listing \ref{lst:language_bools}.
        To make it easy on ourselves, we also add Booleans to the language as the domain $\mathbb{B}=\{\top,\bot\}$.
        However, we maintain, that we only want to trace to contain real numbers as a type, so this means we need to trace away these Booleans, and any operations performed on them.
        Which brings us to the following point: there are no operations in our language.
        Whilst we can validate our if-then-else statement with literal Booleans, it is probably more meaningful to actually add operations to the language.
        To keep things general, we will denote these operations as $Op_\mathbb{X}(s_1,\dots,s_n):\mathbb{Y}$, where $\mathbb{X}$ is the domain of the $n$ inputs $s_1,\dots,s_n$, and $\mathbb{Y}$ is the codomain of the operation.
        To keep things simple, we will assume for now that $Op:\mathbb{X}^n\to\mathbb{Y}$ for any domain $\mathbb{X}$ and codomain $\mathbb{Y}$.
        In Listing \ref{lst:language_bools}, we define three domain-codomain pairs for basic operators: $\mathbb{B}\to\mathbb{B}$ for comparing Booleans, $\mathbb{R}\to\mathbb{B}$ for comparing real numbers, and $\mathbb{R}\to\mathbb{R}$ for arithmetic operators.
        These three categories encapsulate the vast majority of what are considered ``basic operations''.

        \begin{quicklst}[caption=Booleans and operations in the language, label=lst:language_bools, gobble=12]
            !*\textbf{Types:}*!
                $\sigma,\tau\coloneqq\mathbb{B}\|\mathbb{R}\|\sigma\to\tau$

            !*\textbf{Terms:}*!
                $s,t\coloneqq$
                    $\dots$
                  | $\texttt{if }s:\mathbb{B}\texttt{ then }t_\top:\tau\texttt{ else }t_\bot:\tau$
                  | $Op_\mathbb{B}(s_1,\dots,s_n):\mathbb{B}$
                  | $Op_\mathbb{R}(s_1,\dots,s_n):\mathbb{B}$
                  | $Op_\mathbb{R}(s_1,\dots,s_n):\mathbb{R}$
        \end{quicklst}

        With our language expanded, we can now trace away our control flow.
        The resulting cases are shown in Listing \ref{lst:tracing_bool}.
        It should be noted that even though we do not trace away basic operations in general, we can (and in fact have to) trace away operations either on Booleans, or producing Booleans, as we have no Boolean type in our trace.
        This effectively also means, that if we have some operations on real numbers, that only result in eventually producing some Boolean, these operations are omitted from the trace as well.
        If this seems odd, we must remind ourselves that the trace we produce can be interpreted as a ``single-line program'', and any part of the program that only exists to produce values that are irrelevant to the single-line program, should not be in the trace.
        It should also be noted that if we'd want to keep Booleans in our trace for some reason, this would still allow us to trace away control-flow, like the if-then-else statements, as they would still be irrelevant for the single-line program.
        However, what to do with the steps leading up to these would become a little more unclear.
        If the Booleans used by the if-then-else statements in our program are either part of the input or output of the program, they should probably also be present in the trace.
        However, if they are not, then they should probably still be discarded as being irrelevant to the single-line program.

        \begin{quicklst}[caption=Control flow tracing, label=lst:tracing_bool, gobble=12]
            $\text{trace}(\Gamma,\texttt{if }s\texttt{ then }t_\top\texttt{ else }t_\bot)\Rightarrow\texttt{if }\text{eval}(s)\texttt{ then }\text{trace}(\Gamma,t_\top)\texttt{ else }\text{trace}(\Gamma,t_\bot)$

            $\text{trace}(\Gamma,Op_\mathbb{B}(e_1,\dots,e_n):\mathbb{B})\Rightarrow(\text{eval}(Op_\mathbb{B}(e_1,\dots,e_n):\mathbb{B}),\emptyset)$

            $\text{trace}(\Gamma,Op_\mathbb{R}(e_1,\dots,e_n):\mathbb{B})\Rightarrow(\text{eval}(Op_\mathbb{R}(e_1,\dots,e_n):\mathbb{B}),\emptyset)$

            $\text{trace}(\Gamma,Op_\mathbb{R}(e_1,\dots,e_n):\mathbb{R})\Rightarrow(\text{eval}(Op_\mathbb{R}(e_1,\dots,e_n):\mathbb{R}),$
                $\{Op_\mathbb{R}(e_1,\dots,e_n):\mathbb{R}\}\cup\text{trace}(\Gamma,e_1)\cup\dots\cup\text{trace}(\Gamma,e_n))$
        \end{quicklst}

        Finally, we'd like to take a quick look at let-bindings.
        In general let-bindings are can be treated as lambda abstractions that are instantly applied.
        This allows us to streamline them a little more than we were able to with regular lambda abstractions in Listing \ref{lst:tracing}.
        So, we can add let-bindings to our language in Listing \ref{lst:language_let}, and their trace in Listing \ref{lst:tracing_let}.

        \begin{quicklst}[caption=Adding let bindings, label=lst:language_let, gobble=12]
            !*\textbf{Types: }*!$\dots$

            !*\textbf{Terms:}*!
                $s,t\coloneqq$
                    $\dots$
                  | $\texttt{let }s:\sigma\texttt{ in }t:\tau$
        \end{quicklst}

        \begin{quicklst}[caption=Tracing let bindings, label=lst:tracing_let, gobble=12]
            $\text{trace}(\Gamma,\texttt{let }s\texttt{ in }t)\Rightarrow(\text{eval}(\texttt{let }s\texttt{ in }t),\texttt{trace}(\Gamma,s)\cup\texttt{trace}(\Gamma\cup\{\text{eval}(s)\},t))$
        \end{quicklst}

    \subsection{Array Tracing}
        An important part of programming, including the programs where tracing is likely to be used, is arrays and other data structures.
        Tracing data structures like arrays might seem a little more complicated, however the main point is about whether or not we want to trace away arrays.
        For our upcoming example, we first add Arrays to our language from Section \ref{sec:steps} (Listings \ref{lst:language}, \ref{lst:language_bools}, and \ref{lst:language_let}), as shown in Listing \ref{lst:language_array}.
        For now, we will limit our discussion to arrays of real numbers only.
        We will also add operations on arrays, currently just as the operations producing a single value from the array, like indexing or sum ($[\mathbb{R}]\to\mathbb{R}$), and those that produce a new array like mapping a function ($[\mathbb{R}]\to[\mathbb{R}]$).

        \begin{quicklst}[caption=Adding arrays, label=lst:language_array, gobble=12]
            !*\textbf{Types: }*!
                $\sigma,\tau\coloneqq\mathbb{B}\|\mathbb{R}\|\sigma\to\tau\|[\mathbb{R}]$

            !*\textbf{Terms:}*!
                $s,t\coloneqq$
                    $\dots$
                  | $[s_1:\mathbb{R},\dots,s_n:\mathbb{R}]$
                  | $Op_{[\mathbb{R}]}(s):\mathbb{R}$
                  | $Op_{[\mathbb{R}]}(s):[\mathbb{R}]$
        \end{quicklst}
        
        If we were to extend our example from Section \ref{sec:steps}, we might be inclined to not allow arrays in our trace (and keep only real numbers), meaning we would have to trace the arrays away.
        This isn't too complicated.
        When tracing away arrays as terms, we might break them down into their component parts.
        For instance, an array of real numbers, can be traced as just the instantiation of every real number in the array.
        The same goes for operations done to the array: we just trace these operations as they are applied to the individual items in the array, effectively ignoring the fact that these items were in an array in the first place.
        Listing \ref{lst:tracing_array_1} shows what this would look like in the form of our earlier examples.
        Here we have $Op^*_\mathbb{R}:\mathbb{R}$ as the operator we're applying on a single item in an array.
        To clarify, for operations like sum, $Op^*_\mathbb{R}:\mathbb{R}$ would be $s_1+\dots+s_n$, where all these binary additions would end up in the trace as single operations.
        For operations like mapping a function, $Op^*_\mathbb{R}:\mathbb{R}$, would represent the function application on a single item in the array, where we would trace the body of the function like we had traced generic application earlier.
        \cc{This is probably too vague, the notation also falls a bit short here.}

        \begin{quicklst}[caption=Tracing away arrays, label=lst:tracing_array_1, gobble=12]
            $\text{trace}([\Gamma,s_1,\dots,s_n])\Rightarrow([s_1,\dots,s_n],\text{trace}(\Gamma,s_1)\cup\dots\cup\text{trace}(\Gamma,s_n))$

            $\text{trace}(\Gamma,Op_{[\mathbb{R}]}([s_1,\dots,s_n]):\mathbb{R})\Rightarrow(\text{eval}(Op_{[\mathbb{R}]}([s_1,\dots,s_n]):\mathbb{R}),$
                $\text{trace}(\Gamma,Op^*_\mathbb{R}(s_1):\mathbb{R})\cup\dots\cup\text{trace}(\Gamma,Op^*_\mathbb{R}(s_n):\mathbb{R}))$

            $\text{trace}(\Gamma,Op_{[\mathbb{R}]}([s_1,\dots,s_n]):[\mathbb{R}])\Rightarrow(\text{eval}(Op_{[\mathbb{R}]}([s_1,\dots,s_n]):[\mathbb{R}]),$
                $\text{trace}(\Gamma,Op^*_\mathbb{R}(s_1):\mathbb{R})\cup\dots\cup\text{trace}(\Gamma,Op^*_\mathbb{R}(s_n):\mathbb{R}))$
        \end{quicklst}

        We see in Listing \ref{lst:tracing_array_1} a template for tracing almost any operation performed on the array.
        This comes down to knowing that array operations are interested in the individual values of the array, rather than the array as an closed object.
        However, tracing arrays away also comes with some caveats.
        First off, if we have large arrays, our traces will become very large, even if the operations performed on these large arrays are relatively simple.
        Furthermore, it might be unclear in the trace that we were even using arrays in the original program, which depending on your application might be a problem.

        So what if we do not trace away these arrays?
        We would first need to extend the types in our trace with arrays ($[\mathbb{R}]$), however if we wish to keep arrays in our trace this should not be a problem.
        What is more interesting is how we would actually trace these arrays.
        Like the operations on regular real numbers, if we keep arrays in our trace, we keep operations on arrays in our trace.
        This makes tracing arrays really simple as well, as we just repeat arrays and operations on arrays in our trace, this is shown in Listing \ref{lst:tracing_array_2}.
        This does mean that we have no room for individual array items in our trace, except if they actually come out of the array (like with indexing).

        \begin{quicklst}[caption=Keeping arrays in traces, label=lst:tracing_array_2, gobble=12]
            $\text{trace}(\Gamma,[s_1,\dots,s_n])\Rightarrow([s_1,\dots,s_n],\{[s_1,\dots,s_n]\})$

            $\text{trace}(\Gamma,Op_{[\mathbb{R}]}(s):\mathbb{R})\Rightarrow(\text{eval}(Op_{[\mathbb{R}]}(s):\mathbb{R}),\{Op_{[\mathbb{R}]}(s):\mathbb{R}\}\cup\text{trace}(\Gamma,s))$

            $\text{trace}(\Gamma,Op_{[\mathbb{R}]}(s):[\mathbb{R}])\Rightarrow(\text{eval}(Op_{[\mathbb{R}]}(s):[\mathbb{R}]),\{Op_{[\mathbb{R}]}(s):[\mathbb{R}]\}\cup\text{trace}(\Gamma,s))$
        \end{quicklst}

    \clearpage
\section{Automatic Differentiation} \label{sec:ad}
    Tracing is useful in many applications, one of which is Automatic Differentiation (AD).
    Recall how in AD we wish to calculate the derivative of a computer program.
    To do this (in reverse-mode) we wish to calculate the adjoints for the inputs.
    However, to calculate these adjoints, we would first need to calculate the adjoints for the individual computational steps in the program that contribute to an input's sensitivity.
    Of course, it are these steps that are represented in the trace of a program.
    In fact, there is a really close relation between the tapes discussed in Section \ref{sec:bg_ad} and tracing.

    The main difference between the tape used for AD and a regular trace as laid out in Section \ref{sec:tracing}, is the lack of intermediate values in the latter.
    However, provided a trace, we could simply calculate these intermediate values.
    Even better is just storing the intermediate values while we trace a program; this is not really any extra work because these intermediate values are calculated by the tracing function already.
    Consider our trace definition in Listing \ref{lst:traced} as a list of tuples consisting of strings as identifiers and a data constructor denoting the action taken.
    We could just add intermediate values to this structure, but we will soon find this not to be quite enough.
    
    \begin{figure}[htb]
        \centering
        \includegraphics[scale=0.5]{diagrams/forward_example.png}
        \caption{Computational graph of $f(x_1,x_2)\coloneqq x_1+(x_1\times x_2)$}
        \label{fig:forward_graph}
    \end{figure}
    For instance, look at the computational graph in Figure \ref{fig:forward_graph} for $f(x_1,x_2)\coloneqq x_1+(x_1\times x_2)$.
    Now, let us say $x_1=5$, and $x_2=3$, and trace it using the method from Section \ref{sec:tracing}.
    This gives us the trace as \texttt{trace\_result} in Listing \ref{lst:forward_trace}.
    This trace is very straightforward: $x_1$ and $x_2$ are assigned their values, and the multiplication is used in the addition, so it shows up first.
    \begin{haskell}[caption=DSL definition of $f$ and its trace, label=lst:forward_trace, gobble=8]
        f :: Value -> Value -> Expression
        f x1 x2 = ELet "x1" (ELift x1) (
            ELet "x2" (ELift x2) (
                EOp2 Add (ERef "x1") (
                    EOp2 Mul (ERef "x1") (ERef "x2")
                )
            ))

        trace_result :: (TValue, Trace)
        trace_result = (TReal "r2" 20.0, [
            ("x1", TLift (TReal "x1" 5.0)),
            ("x2", TLift (TReal "x2" 3.0)),
            ("r1", TOp2 Mul "x1" "x2"),
            ("r2", TOp2 Add "x1" "r1")
        ])
    \end{haskell}
    Now, let us look at the partial derivatives of $f$ in Equation \ref{eq:reverse_ex}, as we would calculate them using chain rule.
    In Equation \ref{eq:reverse_ex2} we see which calculations we need to perform, we define the partial derivatives or ``adjoints'' of a variable $r_i$ as $\bar{r}_i$.
    We also assume here that the ``seed'' value (the value of $\bar{f}$) is one.
    \begin{equation} \label{eq:reverse_ex}
        \begin{aligned}
            \frac{df}{d\vec{x}}=\nabla f&=\begin{bmatrix}
                \frac{\partial r_2(x_1,r_1)}{\partial x_1}\\
                \frac{\partial r_2(x_1,r_1)}{\partial x_2}
            \end{bmatrix}\tran\\
            &=\begin{bmatrix}
                \frac{\partial x_1}{\partial x_1}+\frac{\partial r_2(x_1,r_1)}{\partial r_1}\cdot\frac{\partial r_1(x_1,x_2)}{\partial x_1}\\
                \frac{\partial x_1}{\partial x_2}+\frac{\partial r_2(x_1,r_1)}{\partial r_1}\cdot\frac{\partial r_1(x_1,x_2)}{\partial x_2}
            \end{bmatrix}\tran\\
            &=\begin{bmatrix}
                1+\frac{\partial r_2(x_1,r_1)}{\partial r_1}\cdot\frac{\partial r_1(x_1,x_2)}{\partial x_1}\\
                0+\frac{\partial r_2(x_1,r_1)}{\partial r_1}\cdot\frac{\partial r_1(x_1,x_2)}{\partial x_2}
            \end{bmatrix}\tran\\
            &=\begin{bmatrix}
                1+1\cdot\frac{\partial r_1(x_1,x_2)}{\partial x_1}\\
                0+1\cdot\frac{\partial r_1(x_1,x_2)}{\partial x_2}
            \end{bmatrix}\tran\\
            &=\begin{bmatrix}
                1+x_2\\
                x_1
            \end{bmatrix}\tran
        \end{aligned}
    \end{equation}
    \begin{equation} \label{eq:reverse_ex2}
        \begin{aligned}
            \bar{f}=\bar{r}_2&=1\\
            \bar{r}_1&=\bar{r}_2\times1\\
            \bar{x}_2&=\bar{r}_1\times x_1\\
            \bar{x}_1&=\bar{r}_2\times1\\
            &+\bar{r}_1\times x_2
        \end{aligned}
    \end{equation}
    With our trace and derivative operations defined, we can now look at how we would get from one to the other.
    It is important to start at the output of the program, and since the trace function we defined in Section \ref{sec:tracing} provides us with the named output, we know where to start on our reverse pass.
    In this case, that would be $r_2$.
    As the final value in the primal calculation is the output of the program, its adjoint will be equal to the adjoint of the program or the seed value.
    This is why Equation \ref{eq:reverse_ex2} posits $\bar{f}=\bar{r}_2$.

    Since we are currently working in reverse execution order, we can just use $\bar{r}_2$ to calculate $\bar{x}_1$ and $\bar{r}_1$ directly.
    It should be reiterated that the trace does not encode any explicit information on the order of operations taken while tracing.
    It is of course a list that was built up one operation at the time, but relying on this forces us to do our reverse pass linearly through the trace, which would prevent some task parallelism opportunities.
    Furthermore, while we can also deduce some order from the naming of the intermediate steps (e.g. $r_1$ was done before $r_2$), we should not do this programmatically, because we wish to reserve parallelism opportunities, but also because some intermediate steps might be hidden in the sub-trace of a map.
    Luckily, we can also discover the ``ancestors'' of any step in the trace by looking at the traced operation.
    For $r_2$ the traced operation was \texttt{TOp2 Add "x1" "r1"}, so we know that for our reverse pass, we next want to look at $x_1$ and $r_1$, as their adjoints (or part of them) rely on the value of $\bar{r}_2$ (which we can also see in Equation \ref{eq:reverse_ex2}).
    For now we will gloss over how we decide which ancestor adjoint to compute first, and just look at the adjoint of $r_1$.
    
    We know that $\bar{r}_1$ is dependent on $\bar{r}_2$, but how exactly is defined by the operation that produced $r_2$, which in this case is addition.
    Now, addition is really simple, as the derivative of addition of two values is the addition of the derivatives of those values.
    See Equation \ref{eq:reverse_add}, where we calculate the adjoint $\bar{r}_1$ and see how this addition just resolves to $1$.
    \begin{equation} \label{eq:reverse_add}
        \begin{aligned}
            \bar{r}_1&=\bar{r}_2\cdot\frac{\partial r_2(x_1, r_1)}{\partial r_1}\\
            &=\bar{r}_2\cdot\frac{\partial(x_1+r_1)}{\partial r_1}\\
            &=\bar{r}_2\cdot\left(\frac{\partial x_1}{\partial r_1}+\frac{\partial r_1(x_1,x_2)}{\partial r_1}\right)\\
            &=\bar{r}_2\cdot(0+1)\\
            &=\bar{r}_2
        \end{aligned}
    \end{equation}

    We can again find the ancestors of $r_1$ by looking at the trace, where we find $x_1$ and $x_2$.
    Let us look at $x_2$ first.
    $\bar{x}_2$ is dependent on $\bar{r}_1$, which we just calculated, but rather than an addition (like $r_2$), $r_1$ is a multiplication.
    We mentioned in Section \ref{sec:bg_ad}, in Equation \ref{eq:dualnumbers}, how the derivative of a multiplication uses both the primal part and the derivative part of a number.
    To get $\bar{x}_2$ we realize (as is visible in Equation \ref{eq:reverse_ex2} as well), that we need the primal value of $x_1$.
    We mentioned before we needed the intermediate values, and this is why.
    Multiplication is not the only operation that requires a primal component, but it is a prime example.
    We see in Equation \ref{eq:reverse_mul} how this adjoint resolves to use the primal component $x_1$.
    \begin{equation} \label{eq:reverse_mul}
        \begin{aligned}
            \bar{x}_2&=\bar{r}_1\cdot\frac{\partial r_1(x_1,x_2)}{\partial x_2}\\
            &=\bar{r}_1\cdot\frac{\partial(x_1\cdot x_2)}{\partial x_2}\\
            &=\bar{r}_1\cdot x_1
        \end{aligned}
    \end{equation}

    Now would also be a good time to quickly reflect on the difference between the tangent (from forward-mode AD) and the adjoint.
    In forward-mode AD, the operation taken to produce some variable, would influence the tangent of that variable.
    This is somewhat intuitive, $r_1$ is a multiplication, and its tangent is $\dot{r_1}=\dot{x_1}\times x_2+\dot{x_2}\times x_1$.
    However, this is not the case for reverse-mode AD.
    In reverse-mode, we see that this information gets passed on to the adjoints of the variable used by the operation, rather than the variable it produced.
    It should be clear why: the tangents denote how the variable is influenced by a change in the inputs, while an adjoint denotes how its corresponding variable influences the outputs.
    It is important to closely observe this, mainly for implementation purposes: we want to calculate (part of) the adjoint before we actually arrive at that step in the trace.
    To calculate $\bar{x}_2$ we need to know what variable $x_2$ was multiplied with (namely $x_1$).
    This means that if we do not want to search through our trace looking for references (to $x_2$ for example) every time, it would be better to calculate (the relevant part of) $\bar{x}_2$ while we still see how it is being used.\\
    This then also bring us neatly to our next conundrum: what if a variable is used multiple times.
    In the example, this goes for $x_1$, something that we have ignored until now.
    The mathematical solution is simple: the partial derivative of a variable that is used multiple times, is just a summation of the adjoints arising from those uses.
    We see this in Equation \ref{eq:reverse_ex2}, where $\bar{x}_1$ is calculated by adding the influence from $r_1$ and the influence from $r_2$ together.
    However, implementation-wise this can be a bit of a hurdle.

    As mentioned, the trace is not in any order.
    This is unlike a typical Wengert list or tape.
    While assuring some order beforehand, or doing topological sort on the computational graph described by the trace, will in large part solve this problem, it also enforces linear execution of the reverse pass.
    And while it is not something we will linger on for now, allowing for concurrency or task parallelism while calculating the derivative might be a nice for a performance boost, and complement the inherent data-parallelism opportunities of array operations.
    So, to solve this, we want to include some form of reference counting.
    During the forward pass we could count how many times each variable is used in the trace.
    Since we need to store intermediate values anyway, keeping a counter for each of these variables seems like little extra work.
    Now, on the reverse pass we can check these reference counters and every time we find part of the adjoint for a variable, we decrement its associated counter.
    If a counter has not reached zero after we have decremented it, we know its adjoint is not yet complete, and we can ignore it for now.
    If it has we can add up all the parts of the adjoint and continue from there.
    This is actually very similar to Kahn's algorithm for topological sorting\cite{kahn1962topological}, except that rather than sorting the graph beforehand, we immediately process the nodes as they become available (have all their incoming adjoints).
    This means we actually do execute the reverse pass in topological order, but by discovering this order as we go it allows us to not strictly do the reverse pass sequentially, something of which we will discuss the merits of further on.
    Provided there is only one output to the program, we know that all reference counters will eventually reach zero, and therefore we are assured we will calculate all adjoints.
    However, this provision is not as clear-cut as it seems.
    Currently our DSL does not really have any room for multiple outputs, and as it is functional does not support any side-effects.
    Instead, to provide multiple outputs, currently the only way is to output an array.
    If we keep arrays in the trace, an array as output would still count as a single value.
    There is a slight discrepancy between the trace and the output if we trace away arrays however: the program will still output an array, but only its individual items are able to be found in the trace.
    This is not really a big problem, since the name of these individual outputs are derived from the name of the full array, but also because it would make little sense to trace away arrays from a program that outputs an array.

    So, we find that our trace needs to be extended with two additional things in the forward pass: intermediate values and reference counters.
    We do this in Listing \ref{lst:forward}, in the data type \texttt{Forward}.
    We also introduce a clone of the \texttt{Traced} data type as \texttt{Forwarded}, as we need to reference the new \texttt{Forward} type in the constructors for maps and vectorized maps.
    We also replace the list structure of \texttt{Trace} with a key-value map.
    This is not strictly necessary, but it allows us to more quickly access the values in the map, while also clearly communicating there is no pre-set order to the trace.
    Each value in a \texttt{Forward} map is a 3-tuple consisting of respectively: the intermediate value, the traced operation performed, and the reference counter for this variable.
    Other than the added reference counting, and saving of intermediate values, the tracing process remains the same as it was in Section \ref{sec:tracing}.

    \begin{haskell}[caption=Forward pass data structures, label=lst:forward, gobble=8]
        data Forwarded
            = FLift TValue
            | FOp0  Op0
            | FOp1  Op1       String
            | FOp2  Op2       String String
            | FMap  [Forward] String
            | FMapV Forward   String

        type Forward = Map String (TValue, Forwarded, Int)
    \end{haskell}

    \subsection{The Reverse Pass}
        As discussed, to facilitate our reverse pass we need both the reference counting and intermediate values.
        Now let us define a function \texttt{reverse} that does the reverse pass.
        This reverse pass should find all the adjoints in the program.
        So, it should take in an object of the \texttt{Forward} type and output a map containing the adjoints.
        In Listing \ref{lst:reverse_def} we define three constructors for adjoints: one for arrays, one for sparse arrays (represented by a single index and the associated value), and one for real values.
        We also define the \texttt{Reverse} type, which will contain these adjoints, and which is returned at the end of the reverse pass.
        The \texttt{Reverse} type maps the names of each part of the calculation to a 2-tuple containing a list of contributions of other adjoints, and its own final adjoint which uses maybe to indicate whether or not it has been calculated yet.

        \begin{haskell}[caption=Definition of the \texttt{Reverse} type, label=lst:reverse_def, gobble=12]
            data Adjoint
                = AArray  [Float]
                | ANull
                | AReal   Float
                | ASparse Int Float

            type Reverse = Map String ([Adjoint], Maybe Adjoint)
        \end{haskell}

        Now before going into precise implementation details we should look at the general picture once more.
        The forward pass provides us with three important components: the final output value of the forward evaluation, the trace on which to do our reverse pass, and the intermediate values we will need to actually calculate everything in the reverse pass.
        First off, the final output value is not actually important for the trace, were it not that it also stores its name in the constructor (for \texttt{TValue}, see Listings \ref{lst:traced} and \ref{lst:language_array}).
        This name points us where to start with the reverse pass, namely the step that produced this output value.

        With our starting point clear, we can now start the reverse pass.
        The programmer will provide some sort of adjoint value (either a real number or an array of them, depending on the output of the regular program), which we will immediately assign to our output value in the reverse pass.
        This makes us ready to actually perform the rest of the reverse pass.

        For any point in the reverse pass the process becomes simple.
        Given some ``current'' point in the computational graph, we look up the adjoint (which should have been established by now) and the forward trace item for this point.
        If the operation in the trace for the current point uses no other values (i.e. has no ``ancestors''), we are done here and return the reverse mapping that contains all the adjoints we have found.
        If the operation does have ancestors, we look at the operation itself to determine how to transform the current point's adjoint for its ancestors.
        If this transformation requires any intermediate values, we can look them up in the forward pass.
        Given the transformed adjoints, we assign these to the adjoint accumulation list for each ancestor.
        We also check if this list now has enough adjoints to match the reference counter in the forward pass.
        If it does not, we are done and can return the reverse mapping.
        However if it does, we add up all the partial adjoints in the list together into the final adjoint and place it in the reverse mapping.
        Then finally, we start the same process for each ancestor of the current node that has its complete adjoint ready.

        Now, let us talk implementation.
        We define two reverse pass functions, \texttt{reverse} and \texttt{resolve}, of which the first will only be a wrapper for the final value and its adjoint to be inserted, and the latter will actually perform the reverse pass.
        Both are shown in Listing \ref{lst:reverse_func}.
        Listing \ref{lst:reverse_func} also introduces two helper functions: \texttt{combineAdjoints} for adding partial adjoints together into the final adjoint of a step in the computational graph, and \texttt{assignAdjoints} for transforming and assigning the adjoints to the ancestors of the current node.
        We will safe the intricate details of \texttt{combineAdjoints} and \texttt{assignAdjoints} for later.
        Finally, we use the \texttt{explore} helper function to try and resolve all the ancestors of the current node as well.

        \begin{haskell}[caption=Definition of the reverse pass functions, label=lst:reverse_func, gobble=12]
            reverse :: String -> Adjoint -> Forward -> Reverse
            reverse s a f = resolve s f $\$$ Map.singleton s ([a], Nothing)

            resolve :: String -> Forward -> Reverse -> Reverse
            resolve s f r = case Map.lookup s r of
                -- Lookup the state of the provided name in the reverse mapping
                -- If its present, but its final adjoint not calculated, we need to check
                -- if we can calculate it
                Just (as, Nothing) -> case Map.lookup s f of
                    -- Find the step taken and the reference counter in the forward pass
                    -- If it is present, check whether or not the adjoint array contains
                    -- items equal to the reference counter (so we know it's all there).
                    Just (fd, c, _) -> if   length as >= c
                                       -- If all partials are present, make the complete
                                       -- adjoint and update the reverse map
                                       then let (a,  r1) = combineAdjoints s f r
                                            -- Then transform and assign the adjoints to
                                            -- the ancestors of this current node.
                                                (r2, sa) = assignAdjoints fd s a f r1
                                            -- Then try all the ancestors as well
                                            in  explore sa r2
                                       -- If we're not ready, just return the current
                                       -- reverse map
                                       else r
                    -- If the named variable isn't present in the forward trace, we've
                    -- got a problem, so we throw an error
                    Nothing         -> error "Variable not in forward trace"
                -- If the adjoint is present, and its final adjoint is already calculated
                -- then we must already be done here, so just return the current reverse
                -- mapping
                Just _             -> r
                -- If the adjoint is missing from the reverse mapping entirely, it's because
                -- we haven't run into it at all yet, so we can also return the reverse mapping
                -- as it is.
                Nothing            -> r
                where
                    -- Applies resolve over a list of ancestors
                    explore :: [String] -> Reverse -> Reverse
                    explore []      r' = r'
                    explore (s':ss) r' = explore ss (resolve s' f r')
        \end{haskell}

        We represent this process on the example program from Figure \ref{fig:forward_graph}, in Figure \ref{fig:reverse_graph}.
        In Figure \ref{fig:reverse_graph} we see the computational graph (forward pass) from Figure \ref{fig:forward_graph} first, as (A).

        Then, with $f=r_2$ as our output value from the forward pass, we can call \texttt{reverse} with some adjoint $a$, the name of $r_2$, and the forward pass we just found to get the reverse graph at (B); this assigns a value to $r_2$ in the reverse map, which we call $r'_2$ in Figure \ref{fig:reverse_graph}.

        As part of the reverse map, $r'_2$ contains two items: a list of partial adjoints (currently only containing $a$), and a final adjoint that has not been calculated yet (so is stored as a \texttt{Nothing}).

        Now we can get into the main loop by calling \texttt{resolve} for $r_2$ with the reverse mapping created by \texttt{reverse}.
        As we can see in graph (C), as we find the partial adjoints $r'_2$ and find that the reference counter in the forward pass is $1=|r'_2|$, we can also call \texttt{combineAdjoints}, which transforms the list of partial adjoints in $r'_2$ to the complete adjoint $a$.

        Then we call \texttt{assignAdjoints} to get graph (D).
        \texttt{assignAdjoints} uses the forward pass to find the ancestors of the current node, in this case $x_1$ and $r_1$, and it also adds the final adjoint of the current node ($r'_2$'s final adjoint is $a$) to their lists of partial adjoints.

        Then \texttt{resolve} on $r_2$ calls \texttt{explore} leading to a \texttt{resolve} call to each of $r_2$'s ancestors, with first up $x_1$ in graph (E).
        However, as we can see from the forward graph (A), we still lack the adjoint from $r_1$ to $x_1$ (highlighted with the red arrow in graph (E)), so we can not resolve $x_1$ yet.

        We leave $x_1$ for later, and this leads us to the resolve call on $r_1$ in graph (F).
        Here we find that $r_1$ does have all its partial adjoints, so we call \texttt{combineAdjoints} to find that the completed adjoint for $r_1$ is also $a$.

        Again, with this adjoint found, we can now call \texttt{assignAdjoints} to bring the adjoint of $r_1$ to its ancestors $x_1$ and $x_2$.
        We will go into how later, but \texttt{assignAdjoints} looks in the forward pass to find that $r_1$ is a multiplication and appropriately transforms $r_1$'s adjoint of $a$ into $a\cdot x_2$ for $x'_1$, and $a\cdot x_1$ for $x'_2$.

        Now in our final steps represented in graph (H), we explore the ancestors of $r_1$, starting with $x_1$.
        We find that we now do have the correct number of partial adjoints for $x_1$, so we can add these together with \texttt{combineAdjoints} to get the final adjoint of $x_1$: $a+a\cdot x_2$.
        Now, after \texttt{combineAdjoints} we call \texttt{assignAdjoints} on $x_1$, however we will find that $x_1$ has no ancestors, which means that $x_1$'s adjoint does not need to be assigned to anything and also that there is no further nodes to explore from $x_1$.

        This means we move back to $r_1$'s explore function, which leads us to $x_2$.
        Here we find again that we can use \texttt{combineAdjoints} to get $x_2$'s final adjoint, and with\\\texttt{assignAdjoints} that $x_2$ has no further ancestors.

        This then moves us back to the end of the explore function in $r_1$, which now finished returns the update reverse mapping to the end of the explore function of $r_2$, which also finishes to return the updated reverse mapping to the original \texttt{reverse} function, and return our reverse pass to the programmer.
        
        \begin{figure}[p]
            \begin{adjustbox}{addcode={\begin{minipage}{\width}}{\caption{%
                Diagram representing the reverse pass process on the example from Figure \ref{fig:forward_graph}.
                }\label{fig:reverse_graph}\end{minipage}},rotate=90,center}
                \includegraphics[scale=.6]{diagrams/reverse_example.png}
            \end{adjustbox}
        \end{figure}

        This gives us a global overview of how the reverse pass can be implemented, using the forward pass/trace we have discussed.
        Now there are a couple of questions that remain:
        \begin{itemize}
            \item Can we always add adjoints together?
            \item How do different operations differentiate?
            \item How do we maintain data-parallelism in array operations?
            \item How do we implement task-parallelism on the reverse pass? 
        \end{itemize}
        In the following subsections we will get to all these questions.

    \subsection{Combining Adjoints}
        To go from a list of partial adjoints to a single combined adjoint is not quite as trivial as just adding all together.
        As mentioned before, in Listing \ref{lst:reverse_def}, we defined three types of adjoints, an array, a sparse array, and a real number adjoint.
        The real number adjoint is not complicated, it is the default adjoint and they can be freely added together.
        The difficulty comes in with the array adjoints.
        These adjoints are produced by operations on arrays.
        It will help us to realize now that the adjoint of an array with length $n$, will also have length $n$.
        In reality the adjoint array is no more than an array of adjoints for each item in the original array.

        Knowing this we can start to discover how to add these adjoints together.
        Let us start by discussing adding the sparse and non-sparse array adjoints, as they are almost as simple as adding two real adjoints.
        Since we only add these together as partial adjoint of a single step in the computational graph, we know that the array adjoints whether sparse or not will always have the same length when added together.
        This means we can just add these arrays of adjoints together elementwise, substituting $0$ for all undefined items in the sparse array.

        Only if we add two sparse arrays together, we might need to look up the length of the original array in the forward pass to know the length of these sparse arrays (or we could extend the sparse array constructor to allow for multiple defined items).

        Now the most interesting part comes when we want to add an array adjoint to a real adjoint, or vice versa.
        Namely, there are two ways of doing this, which means that a binary adjoint addition operator is not associative.
        For example, say we find ourselves folding the values in some array $a$ to some real value $b$.
        In the reverse pass, the adjoint of $b$ will be a real adjoint, yet the adjoint of $a$ will have to be an array adjoint.
        Now provided that $a$ has some partial array adjoint (or is represented by an adjoint array filled with zeroes) we must think of a way to add $b$ to the appropriate items.
        Now what these appropriate items in $a$ are, and how $b$'s adjoint needs to be transformed, are of course dependent on the function we fold over array $a$.

        However, we will find that fold takes in an anonymous function, meaning that we need a special case for fold anyways.
        Only for our unary sum operator we perform a fold that is pre-programmed.
        We know that the sum's result is influenced once by each array item, and from the differentiation rules of addition we know that the adjoint of the sum's result is just moved to its ancestors without transformation or touching intermediate values.
        This means that we can just just add the sum's adjoint to each array item.

        The only other way to get from an array to a real value in our language is by using array indexing.
        However this is very simple as well, as indexing does not change anything to the adjoint either.
        Since indexing refers to a specific item in the array, our partial adjoint for the array becomes a sparse array with the incoming adjoint on the indexed position.

        We see the implementation of the helper function \texttt{combineAdjoints} in Figure \ref{lst:combine}, where we also define a binary adjoint addition operator \lstinline{(<+)} to do most of the heavy lifting.
        It should be reiterated that the \lstinline{(<+)} operation adding a real to an array only works because it will only be called by the sum operator.

        \begin{haskell}[caption=Adjoint combination operator and adjoint summation function, label=lst:combine, gobble=12]
            (<+) :: Adjoint -> Adjoint -> Adjoint
            (<+) (AArray as)   (AArray    bs) = AArray (zipWith (+) as bs)
            -- Note: this works because it is only called by sum
            (<+) (AArray as)   (AReal     bs) = AArray (map (+ b) as)
            (<+) (AArray as)   (ASparse i bs) =
                let bs = drop i as
                in  AArray (take i as ++ (b + head bs) : tail bs)
            (<+) (AReal  a)    (AArray    bs) = AReal (a + sum bs)
            (<+) (AReal  a)    (AReal     b)  = AReal (a + b)
            (<+) (AReal  a)    (ASparse _ b)  = AReal (a + b)
            (<+) (ASparse i a) (AArray    bs) = 
                let as = drop i bs
                in  AArray (take i bs ++ (a + head as) : tail as)
            (<+) (ASparse _ _) (AReal     _)  = error "Cannot combine sparse and real
                adjoints, because the length of the sparse adjoint is unknown."
            (<+) (ASparse _ _) (ASparse _ _)  = error "Cannot combine two sparse adjoints,
                because the length of the sparse adjoints are unknown."

            combineAdjoints :: String -> Forward -> Reverse -> (Adjoint, Reverse)
            combineAdjoints s f r =
                -- Get the partial adjoints and combine them together
                let (as, _) = r Map.! s
                    a       = foldr (<+) empty as
                -- And add the completed adjoint to the reverse map
                in  (a, Map.insert s (as, Just a) r)
                where
                    -- Provide an empty identity element
                    empty :: Adjoint
                    empty = case getValue s f of
                        -- Check the intermediate value of s to reveal the right identity element
                        (FArray _ xs) -> AArray (replicate (length xs) 0.0)
                        (FReal  {})   -> AReal  0.0
                        _             -> error "Type mismatch in combineAdjoints/empty"
        \end{haskell}

    \clearpage
    \subsection{Differentiating Operations}
        The actual differentiation rules are applied in the \texttt{assignAdjoints} function, where we take the complete adjoint of a node in the computational graph, transform it according to the operation performed in that node, and then add it as a partial adjoint to its ancestors.
        Its especially the transformation that means we need to write different differentiation rules for almost every operation.

        Let us start with the easy part, the unary and binary mathematical operators.
        In our DSL these are: addition, multiplication, subtraction, and sine.
        Recall that the rules for addition and subtraction are similar, they just transform homomorphically, which means that any adjoint is just ``passed'' to their ancestors without any additional transformation.
        See the examples in Equation \ref{eq:diff_add}.

        \begin{equation} \label{eq:diff_add}
            \begin{aligned}
                \frac{d(x+y)}{dz}&=\frac{dx}{dz}+\frac{dy}{dz}\\
                \frac{d(x-y)}{dz}&=\frac{dx}{dz}-\frac{dy}{dz}
            \end{aligned}
        \end{equation}

        Multiplication and sine are slightly more complicated, both requiring some intermediate value to compute the derivative.
        We have gone over multiplication before, we multiply the incoming adjoint with intermediate value of the other side of the multiplication.
        See the example in Equation \ref{eq:diff_mul}.

        \begin{equation} \label{eq:diff_mul}
            \frac{d(x\cdot y)}{dz}=y\cdot\frac{dx}{dz}+x\cdot\frac{dy}{dz}
        \end{equation}

        The derivative of a sine operation is the cosine on the intermediate value, see Equation \ref{eq:diff_sin}.

        \begin{equation} \label{eq:diff_sin}
            \frac{d(\sin x)}{dy}=\cos x\cdot\frac{dx}{dy}
        \end{equation}

        These are the mathematical operations that we may encounter in the trace.
        Recall that we traced away Booleans, so we do not have to worry about comparison operators.
        With these derivatives cleared up we can now program them into \texttt{assignAdjoints}.
        It is also useful to remember that \texttt{assignAdjoints} should return both the updated reverse mapping, and a list containing the ancestors of the provided node.
        We see the implementation in Listing \ref{lst:assign_simple}.
        This uses another helper function called \texttt{addAdjoint}, which simply just adds the calculated partial adjoint to the list of partial adjoints of the relevant ancestor in the reverse mapping.

        \begin{haskell}[caption={Defining \texttt{assignAdjoints} for sine, addition, subtraction, and multiplication.}, label=lst:assign_simple, gobble=12]
            assignAdjoints :: Forwarded -> String -> Adjoint -> Forward -> Reverse
                -> (Reverse, [String])
            -- Adjoint function for all unary operators (now only showing sine)
            assignAdjoints (FOp1 op s1) _ a f r = case (op, a) of
                (Sin, AReal a') -> case getValue s1 f of
                    -- Take the intermediate value and assign the adjoint to s1, also return
                    -- the list of ancestors: s1
                    (FReal _ x) -> (addAdjoint s1 (AReal $\$$ a' * cos x) r, [s1])
                    _           -> error "Type mismatch in assignAdjoints/FOp1/Sin"
                _               -> error "Type mismatch in assignAdjoints/FOp1"
            
            -- Adjoint function for all binary operators
            assignAdjoints (FOp2 op s1 s2) _ a f r = case (op, a) of
                -- Just add the adjoint to both ancestors
                (Add, _)        -> (addAdjoint s1 a (addAdjoint s2 a r), [s1, s2])
                -- For multiplication, get the intermediate values first
                -- we also deconstruct the adjoint, knowing it is a real number because
                -- (sin x) would return a real number.
                (Mul, AReal a') -> case (getValue s1 f, getValue s2 f) of
                    (FReal _ v1, FReal _ v2) ->
                        (addAdjoint s1 (AReal (a' * v2)) (addAdjoint s2 (AReal a' * v1) r), 
                        [s1, s2])
                    _                        ->
                        error "Type mismatch in assignAdjoints/FOp2/Mul"
                -- Similar to addition, only s2 gets a negative adjoint
                (Sub, AReal a') -> (addAdjoint s1 a (addAdjoint s2 (AReal -a') r), [s1, s2])
                _               -> error "Type mismatch in assignAdjoints/FOp2"
        \end{haskell}

        Another adjoint we should quickly cover is that of array indexing.
        As mentioned before, nothing happens to an intermediate value when it is indexed, so its adjoint will not be transformed either.
        While obvious, the adjoint of the indexed value should only be added to the that specific index in the array's adjoint.
        This is where our sparse adjoint comes in, where we represent a single item in the array without storing anything extra.
        We can see how it is used by \texttt{assignAdjoints} for indexing in Listing \ref{lst:assign_idx}.

        \begin{haskell}[caption={Defining \texttt{assignAdjoints} for the indexing operation}, label=lst:assign_idx, gobble=12]
            assignAdjoints :: Forwarded -> String -> Adjoint -> Forward -> Reverse
                -> (Reverse, [String])
            assignAdjoints (FOp1 op s1) _ a f r = case (op, a) of
                -- Add the sparse array and return the name of the array as ancestor
                (Idx i, AReal a') -> (addAdjoint s1 (ASparse i a') r, [s1])
                $\dots$
            $\dots$
        \end{haskell}

        % NOTE: While an implementation detail, FJoin (also unmentioned in the forward pass) has a simple adjoint too, used by FFold and FFoldV

        \subsubsection{The reverse pass on map operations and function closures}
            Unsurprisingly, the reverse pass is slightly more in-depth on array operations, like map.
            For a regular non-vectorized map, this is still fairly easy to wrap our heads around.
            Especially if we remember that our forward-pass provides us with the following constructor for such maps:
            \lstinline[language=haskell]{FMap [Forward] String}.
            Here we store every application of the mapped function to our array as its own sub-trace.
            Now it becomes easy to realize that the simplest way to reverse-pass over this mapping is just to call the \texttt{reverse} function on each of these forward sub-traces.
            This then provides us immediately with the adjoint of each original array item, which we can combine into an array adjoint for the original array.
            We give this array adjoint as a partial adjoint to the original array (as it is the ancestor of the mapping operation).
            It should be explicitly stated that mapping the \texttt{reverse} function reveals the derivative of a map: another map, but in reverse.

            However, there is a slight obstacle yet to overcome; to do with the mapped function.
            What do we do if the function that is mapped over the array uses variables from anywhere outside the actual array?
            Recall that the functions that are mapped are lambda expressions that have access to any variables in the environment at the function's definition.
            As an example, we can see this expressed in Figure \ref{fig:map_graph}, where some mapped operation on an array $[b]$ uses a variable $a$ to produce the array $[c]$.
            While the partial adjoints for this variable are computed during the reverse pass over every item, they are only stored in the reverse mapping for the particular item.
            And if we extract only the adjoint for the original array item from this reverse mapping, we would throw away these partial adjoints, making it possibly impossible for us to calculate the full reverse-pass of the program, as we will not be able to calculate the adjoint for these outside variables.
            We find that these variables, like $a$ in \ref{fig:map_graph} are ``unofficial ancestors'' of the mapping operation.
            They are used by the mapping operations, but can be a little hard to find, as they are hidden in the sub-traces.

            \begin{figure}[htb]
                \centering
                \includegraphics[width=0.6\textwidth]{diagrams/map_example.png}
                \caption{Example of an unofficial ancestor to an array operation}
                \label{fig:map_graph}
            \end{figure}

            Luckily for us the main difficulty with this problem is noticing it.
            Now we know that we have to extract this data into the main reverse pass, we can simply merge the item's reverse pass with the main one.
            We only need to be wary of extracting the partial adjoints for the original map, and combining them together, so we do not overcount the number of partial adjoints against the reference counter of the original array (that in the forward pass only got increased by one for use in the map operation.)
            We see the whole process of finding the adjoint of a map in Listing \ref{lst:assign_map}.

            \begin{haskell}[caption=Implementation of \texttt{assignAdjoints} for the map operation, label=lst:assign_map, gobble=16]
                assignAdjoints :: Forwarded -> String -> Adjoint -> Forward -> Reverse
                -> (Reverse, [String])
                $\dots$
                assignAdjoints (FMap fss s1) s a _ r =
                    let (as, r', ss) = reverseMap fss 0
                        -- Fold sparse adjoints into single array adjoint
                        a' = foldl (<+) (AArray $\$$ replicate (length fss) 0.0) as
                    in  (addAdjoint s1 a' r', Set.toList ss)
                    where
                        reverseMap :: [Forward] -> Int -> ([Adjoint], Reverse, Set String)
                        reverseMap []     _ = ([], r, Set.empty)
                        reverseMap (f:fs) i =
                            let s'  = s ++ '!' : show i
                                rx  = reverse f s' (indexAdjoint i a)
                                -- Extract the array item for 
                                ax  = toSparse i $\$$ fst $\$$ combineAdjoints s' f rx
                                -- Remove items from the original and destination array from this reverse pass
                                -- and add the array items partial adjoint
                                rx' = Map.delete s' (Map.delete (s1 ++ '!' : show i) rx)
                                -- Find the results of the rest of the map
                                (axs, rxs, sxs) = reverseMap fs (i + 1)
                            in  (
                                -- Add this sparse partial to the list
                                ax : axs,
                                -- Add rx' to the main reverse pass
                                Map.unionWith unionReverse rxs rx',
                                -- Add relevant keys to the set of ancestors
                                Set.union sxs $\$$ Map.keysSet rx'
                            )
                        toSparse :: Int -> Adjoint -> Adjoint
                        toSparse i (AReal a') = ASparse i a'
                        toSparse _ _          =
                            error "Type mismatch in assignAdjoints/FMap/toSparse"
            \end{haskell}

            Now with the nonvectorized map taken care of, we can move on to the vectorized map.
            While the idea of finding the adjoint to the map remains the same of course, we now find ourselves with a new problem: a lack of intermediate values.
            Recall that when tracing a vectorized map, we only stored the trace of a single item in the array.
            This was possible because, as we had found the mapped function would not branch, the trace would be the same for each item.
            Now, provided that we did store all intermediate values for this mapping operation, we can perform the same reverse map for the vectorized trace, as with the nonvectorized trace.

            \paragraph*{On Parallelism}
                It should be clear that there is a major inherent parallelism opportunity for each of the mapping operations.
                For the vectorized map this is very clear: data parallelism.
                This is true in evaluation, and is still true in the reverse-pass.
                Namely as the reverse pass is the same for all items, we can use a data parallel mapping function to run the reverse pass in a vectorized manner as well.

                For the nonvectorized map we cannot be sure the operations are the same for each item.
                This means that data parallelism is off the table, however we can still use task parallelism.
                This would not be hard to implement as each item has its own reverse call which can easily be run task parallel.
                Of course, one would need to exercise some caution with the collection of the partial adjoints and reverse maps, but this is not a new problem.
                The combination of the partial adjoints would still need to happen sequentially.

                We will talk more about parallelism when we talk about the reverse map on folds.

        \subsubsection{The reverse pass on fold operations}
            Hello, world!



    \section{Conclusion}
    In this thesis, we set out to find a way to preserve data-parallelism through a reverse-mode automatic differentiation approach using tracing.
    We did so mainly by defining a domain-specific language (DSL) for Haskell, and implementing tracing and automatic differentiation on that DSL.

    We found our first obstacle in the tracing, despite or maybe even because of its ubiquity, it was hard to find a proper definition that fit our use case.
    In fact, a formal definition of tracing seemed to be largely absent from literature, excluding some specialized definitions.
    So we first had to start by defining tracing, at least enough for our use case.
    We came to two insights here.
    First, we can decide on what and how to trace a program, by deciding what data types and data structures we wish to keep in the trace, and which we would rather remove or ``trace away''.
    Second, a uniform way of tracing is unlikely to exist, at least on a higher-level of programming.
    This would be due to many operations that may require special case implementations, some even dependent on the types we wish to keep in the trace.
    So while a completely formal universal definition of tracing was off the table, we still managed to create two logic assertions that, although quite broad still, would help us implement tracing on the DSL we wished to explore.
    
    With the tracing worked out, we could now continue to the automatic differentiation.
    Reverse-mode AD exists of two phases, a forward pass and a reverse pass.
    The forward pass would be largely covered by tracing, however it needed a couple adjustment for it to be useful.
    First off, the trace was lacking intermediate values of the program, which would be needed for the reverse pass.
    Recalculating these values during the reverse pass would be very inefficient, so instead we modified the trace to also store intermediate values.
    The second problem was a little more tricky: while the trace would provide us with all the operations done, including a way to reverse pass through the trace by reference names, it did not provide the full structure of the computational graph.
    To be more precise: while calculating the adjoint for some node $a$ on the computational graph, we could not be sure whether this was the only edge leading to $a$, or if there were others.
    This is problematic, because to continue the reverse pass from $a$ to its ancestors, we would need to be sure the adjoint for $a$ was calculated correctly.
    Of course, this could be easily solved by introducing some sorting algorithm to the trace before the reverse pass, however this would force sequential execution of the reverse pass.
    We wanted to avoid this as to leave room for task parallelism in the reverse pass: where we could introduce multiple sub-tasks from a node in the computational graph that has multiple ancestors.
    We managed to solve this problem by drawing inspiration from Kuhn's topological sort algorithm.
    First, we add reference counters to each node in the computational graph, for which we do the counting during the forward pass.
    Then on the reverse pass, when we have a calculated adjoint we wish to assign to an ancestor, we just add it to a list of partial adjoints for that ancestor.
    Now we can check the length of that list of partial adjoints to the reference counter associated with that node, and if they are equal, we would know we had collected all the partial adjoint and could calculate the complete adjoint for that node.
    Using this method, we can enforce an implied topological sort without enforcing that sort beforehand, which allows us the use of task parallelism.
    
    Now using this carefully constructed forward pass, we would be able to do our reverse pass quite easily.
    However, we still had one problem left to deal with: closures.
    Since our DSL only allowed for lambda functions as closures, we already made sure to capture any relevant information in the environment on the forward pass.
    However, this meant that in the reverse pass our lambda functions could call on variables outside the function scope, variables that needed to be updated with relevant adjoints as well.
    For regularly applied functions this was no problem, after all the operations done by that applications would just appear in the trace, leaving no opaque closures for us to deal with.
    However, when we introduced arrays to our DSL, and array operations like map, generate, and fold, we ran into the opaqueness of these functions.
    While the function themselves where traced away into sub-traces for these array operations, these array operations themselves would not clearly display any variables outside the function scope as ancestors.
    When we would pass over them in the reverse pass, and use an independent reverse pass to calculate through the sub-traces, the adjoint contributions for variables irrelevant to the input array would be lost.
    Clearly this is unacceptable, because this would mean missing contributions, and incalculable adjoints for these informal ancestors.
    Luckily the main problem here was identifying it, as it could easily be remedied. 
    We simply combine these reverse passes over the sub-traces with the main reverse pass, such that all contributions are saved.
    The main takeaways from the reverse pass implementation were the insight we gained into closure, and the reverse pass over data-parallel array operations.
    With our forward pass storing the nature of these array operations in sub-traces, we could leverage these sub-traces to do the reverse pass.
    We found that for the operations included in our DSL, we could leverage data-parallelism on the reverse pass in the same places data-parallelism was used in regular execution.
    Similarly, for any point in the regular execution where we could implement task parallelism, we could also do so in the reverse pass.

    With all this done, we ended up with a reverse-mode AD algorithm that would work on our entire DSL.
    While this DSL was not that impressive in scope, it did show us a clear way to use tracing for automatic differentiation.
    Furthermore, it showed us how we could maintain data parallelism in the reverse pass.
    We also found that the performance of our implementation left some room for improvement.
    However, by either letting go of purity for the reverse mapping and using more efficient data structures, we might be able to reach a more desirable time complexity.

    \subsection{Future work}
        Quite some work still remains.
        Such as an expansion to our tracing definition.
        This could either be an exploration into more complex type structures, or formalization of the definition we started.
        Especially with a formalized definition, one might be able to reach some more interesting correctness proofs than the two assertions we found.
        However, the question remains whether tracing can be formalized; we noted that we need special handling of a lot of cases, which might not be neatly formalizable.

        Furthermore, we left the actual implementation of our AD algorithm at a high-level DSL, which is quite different from actual implementation on a lower level or GPU programming.
        While our findings should carry over regardless, it would be most interesting to see what kind of performance improvements can be gained by using data and task parallelism in the reverse pass for AD.
        Such an implementation should probably also include real parallelism, rather than parallelism hinted to by operations that could be parallelized.

        In a similar vein, we probably would want to expand the operations of our DSL, to create more definitions for tracing and AD on specific operations.
        Plenty of parallel array operations remain, including scan, scatter, and gather to name just a few.
        There are also still types structures yet to explore, and types in general that may need to be traced (away).

        Finally, but perhaps most importantly, there is the issue of higher-dimensional arrays.
        Our implementation in this paper dealt only with 1-dimensional arrays, which allowed us to sidestep complex situations, like multidimensional folds.
        While we can reasonably assume the theory laid out in this work should hold in higher dimensions as well, as there is nothing particularly special about higher-dimensional array operations, we could also imagine it becoming a difficult task to implement.


    \clearpage
    \bibliographystyle{unsrt}
    \bibliography{references}
    \clearpage
    \appendix
    \section{ADT Evaluation} \label{sec:eval}
    In Section \ref{sec:steps}, we introduced an extended lambda calculus.
    In this section we will quickly go over how an evaluator function for this ADT would look like in Haskell.
    We define our evaluator function in Listing \ref{lst:eval}, using the definitions of \texttt{Expression}, \texttt{Value}, and \texttt{Environment} from Listing \ref{lst:language}.

    \begin{haskell}[caption=ADT Evaluator, label=lst:eval, gobble=8]
        eval :: Environment -> Expression -> Value

        eval n (EApply e1 e2) =
                -- Evaluate e1 and e2 first
            let v1 = eval n e1
                v2 = eval n e2
            in  case v1 of              
                    -- Only apply v1 to v2 if v1 is a function as expected
                    VFunc f -> f v2
                    _       -> error "Type mismatch in eval/EApply"

        eval n (EIf e1 e2 e3) =
            -- Evaluate e1 as the condition of the if-then-else statement
            case eval n e1 of
                -- If e1 evaluates to true, evaluate e2
                VBool True  -> eval n e2
                -- Otherwise, evaluate e3
                VBool False -> eval n e3
                _           -> error "Type mismatch in eval/EIf"

        -- For abstractions, we return the function by moving the evaluation into the body.
        -- Where we insert the anonymous value x into the environment as it was when the
        -- function was defined.
        eval n (ELambda s1 e1) = VFunc !*\$ \textbackslash*!x -> eval (insert s1 x n) e1

        eval n (ELift v1) = v1

        eval n (EOp2 op e1 e2) =
                -- Evaluate e1 and e2 first
            let v1 = eval n e1
                v2 = eval n e2
                -- This case syntax allows us to select for the right op with the right
                -- value types at the same time.
            in case (op, v1, v2) of
                (Add, VFloat a, VFloat b) -> VFloat !*\$*! a +  b
                (Equ, VBool  a, VBool  b) -> VBool  !*\$*! a == b
                (Equ, VFloat a, VFloat b) -> VBool  !*\$*! a == b
                (Mul, VFloat a, VFloat b) -> VFloat !*\$*! a *  b
                (Neq, VBool  a, VBool  b) -> VBool  !*\$*! a /= b
                (Neq, VFloat a, VFloat b) -> VBool  !*\$*! a /= b
                _                         -> error "Type mismatch in eval/EOp2"

        -- Resolving references means getting the value from the environment by name.
        eval n (ERef s1) = n ! s1
    \end{haskell}
\end{document}