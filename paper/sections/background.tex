\section{Related Work}
    \subsection{Parallel Array Programming}
        Array algorithms in imperative programming often heavily rely on loops.
        While there is no problem in making this efficient through multi-core parallelism, imperative parallel array code often lacks in clarity and may be hard to program or comprehend.

        Instead, the functional paradigm allows us to write code that is both more elegant and data-focussed than imperative code.

        A great starting point for our history in functional parallel array programming, is with the 1995 paper by Aditya, Arvind, Augustsson, Maessen, and Nikhil, ``Semantics of pH: A parallel dialect of Haskell'' \cite{maessen1995semantics}.
        As the title suggests, this paper introduces a parallel programming dialect to Haskell.
        The authors suggest it was their attempt at bringing together the functional lazy-evaluation community of Haskell together with the ``dataflow'' community.
        
